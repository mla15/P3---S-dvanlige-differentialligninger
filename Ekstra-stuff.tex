\chapter{Ekstra-stuff}
\begin{definition}[]
Lad $\textbf{f}$ være en funktion og $a \in \mathbb{R}^n$ være et punkt. Er $\textbf{f}$ differentiabel i $a$, er dens afledte givet ved $\textbf{J}_f({a})$. I dette tilfælde siges den lineære afbildning givet den Jacobianske-matrix at være den bedste lineære approksimation af $\textbf{f}$ til $a$.

\hfill \break
Denne lineære approksimation skrives:

$$\textbf{f}(x)=\textbf{f}(a)+ \textbf{J}_f({a})(x-a)+ o(\left. \Vert \left. x-a \right. \Vert \right.),$$ 
for $x \to a$.

\end{definition}

\begin{definition}[Hyperbolsk ligevægtspunkt]
%asfAFASF
\end{definition}

\begin{Example}

$$\textbf{F(x)} =
\begin{bmatrix}
x_1 x_3^2 -e^{x_1 x_2} \\
x_2 - \frac{1}{x_3} \\
x_1 x_2 x_3
\end{bmatrix} \Rightarrow
J_{f}(x) =
\begin{bmatrix}
    x_3^2 - x_2 e^{x_1 x_2} & -x_1 e^{x_1 x_2} & 2x_1 x_3 \\
    0 & 1 & \frac{1}{x_3^2}\\
    x_2 x_3 & x_1 x_3 & x_1 x_2
\end{bmatrix}$$

\end{Example}
\subsection{Mikkel}

\subsection{Lipschitz-kontinuitet}
Lipcshitz-kontinuitet er første skridt i at bevise eksistens- og entydighedssætningen for løsninger til differentialligninger?
\begin{definition}[Lipschitz-kontinuitet]
Givet en funktion, $f: (t,\vec{x})\in R \subseteq \mathbb{R}^{m+1} \rightarrow f(t,\vec{x}) \in \mathbb{R}^m$, kaldes Lipschitz-kontinuert, hvis $\exists L \geq 0$, så:
$$|f(t,\vec{x_1})-f(t,\vec{x_2})|\leq L|\vec{x_1}-\vec{x_2}|$$
\end{definition}

\begin{Example}\textbf{Lipschitz-kontinuitet}\hfill \break
\textnormal{Hvis vi lader $\mathbb{R}^2$ være det réelle plan og betragter funktionen $f(t,x)=t^5+4x$, så kan vi undersøge om $f$ er Lipschitz-kontinuert på $\mathbb{R}^2$:} \hfill \break
$$|f(t,x_1)-f(t,x_2)|=|(t^5+4x_1)-(t^5+4x_2)|=|4x_1-4x_2|=4|x_1-x_2|$$ \hfill \break
\textnormal{Det ses nu, at funktionen opfylder en Lipschitz betingelse på $\mathbb{R}^2$ med $L=4$.}
\end{Example}
\begin{lemma}{}{}
Lad $D$ være et lukket rektangel $R=(a,b)\times(c,d)$ eller en uendelig strimmel $S=(a,b)\times(-\infty,\infty)$ på planet $\mathbb{R}^2$. $f$ er Lipschitz-kontinuert på $R$ eller $S$ hvis:

\begin{enumerate}
    \item $f:D \rightarrow \mathbb{R}$
    \item $\frac{\partial f}{\partial y}$ eksisterer og er kontinuert
    \item Der eksisterer en konstant $K\geq 0$ sådan, at $|\frac{\partial f}{\partial y}(t,y)|\leq K$
\end{enumerate}

Dermed er $K=L$
\end{lemma}
\begin{proof} \hfill \break
Ved brug af fundamental calculus gælder, at for alle $(t,y_1),(t,y_2) \in D$ er
\begin{align}
 f(t,y_1)-f(t,y_2)=\int_{y_2}^{y_1}\frac{\partial f}{\partial y} dy \notag
\end{align}
og derved
\begin{align}
 |f(t,y_1)-f(t,y_2)|&=|\int_{y_2}^{y_1}\frac{\partial f}{\partial y} dy| \notag \\
&\leq |\int_{y_2}^{y_1}|\frac{\partial f}{\partial y}| dy| \notag\\
&\leq |\int_{y_2}^{y_1}K| \notag \\
&=K|y_1-y_2| \notag
\end{align}
hvilket giver $K=L$ ifølge definitionen af Lipschitz-kontinuitet. \qedhere
\end{proof}
\hfill \break

\begin{Example}\hfill \break
\textnormal{Betragt funktionen $f(t,y)=t^5+\tan^{-1}(y)$. Hvis vi benytter Lemma 2.3.1 får vi:} \hfill \break
\begin{align}
\frac{\partial f}{\partial y}=\frac{1}{1+y^2} \notag
\end{align}
\textnormal{Det ses hurtigt, at brøken $\frac{1}{1+y^2}$ altid vil være mindre end eller lig med 1 for alle $(t,y)\in \mathbb{R}^2$, altså:}
\begin{align}
 \frac{1}{1+y^2}\leq 1, \forall(t,y) \in \mathbb{R}^2 \notag
\end{align}
\textnormal{Dermed ses, at $L=1$, og $f$ er Lipschitz-kontinuert på $\mathbb{R}^2$.}
\end{Example}

\subsection{Banachs fixpunktssætning}

\begin{definition}[Metrisk rum]
Et metrisk rum er et ordnet par $(M,d)$, hvor $M$ er en mængde og $d$ er en metrik $d:M\times M \rightarrow \mathbb{R}_+$, $d(x,y)=|x-y|$, hvorom der gælder for ethvert $x,y,z \in M$: 
\begin{enumerate}
    \item $d(x,y)=0 \Leftrightarrow x=y$ 
    \item $d(x,y)=d(y,x)$
    \item $d(x,z)\leq d(x,y)+d(y,z)$
\end{enumerate}
$d$ kaldes også for afstandsfunktionen.
\end{definition}

\begin{definition}[Cauchyfølger i Metriske rum]
En talfølge $\{x_n\}$ i et metrisk rum $(M,d)$ siges at være Cauchyfølge, hvis der for enhver positiv tolerencegrad $\varepsilon >0$ eksisterer $N \in \mathbb{N}$, så der for alle $m,n \geq N$, $m,n \in \mathbb{N}$ gælder, at: 
$$|x_n-x_m|<\varepsilon$$
\end{definition}

\begin{definition}[Fuldstændigt metrisk rum]
Et metrisk rum $(M,d)$ siges at være fuldstændigt, hvis alle Cauchyfølger i $M$ har en grænse, der også er i $M$. Altså hvis alle Cauchyfølger i $M$ konvergerer i $M$.
\end{definition}

\begin{definition}[Kontraktion og Fixpunkt]
Lad $(M,d)$ være et metrisk rum. En afbildning $F:M\rightarrow M$ kaldes en kontraktion, hvis der eksisterer en Lipschitz-konstant $0 \leq L < 1$, sådan at:
$$|F(x_1)-F(x_2)|\leq L|x_1-x_2|$$
Et punkt $x \in M$ er et fixpunkt for $F$, hvis $F(x)=x$ 
\end{definition}

\begin{mytheo}{Banachs fixpunktssætning}
LLad $(X,d)$ være et fuldstændigt metrisk rum og $F:X\rightarrow X$ være en kontraktion. Så har $F$ et unikt fixpunkt.
\end{mytheo}

\begin{proof}
Først viser vi unikheden af fixpunktet. Antag at der eksisterer $a,b \in M$, så $F(a)=a$ og $F(b)=b$, så antyder Definition 2.12, at \hfill \break
$$0\leq d(a,b)=d(F(a),F(b))\leq Ld(a,b), (1-L)|y_1-y_2|\leq 0$$
hvilket betyder at $d(a,b)=0$ og at $a=b$ \hfill \break
\hfill \break
Vi konstruerer nu sådan et fixpunkt. Betragt talfølgen ${y_n}$, hvor $y_1$ er arbitrær og $y_n:=F(y_{n-1})$ for ethvert $n\geq 2$. Vi ønsker nu at vise to ting:\hfill \break
 ($i$) Talfølgen er Cauchyfølge i M og dermed konvergerer mod et $y_\infty$, da vi antog at $M$ var fuldstændigt metrisk rum.\hfill  \break
($ii$) $y_\infty$ er et fixpunkt for $F$. \hfill \break
Vi starter med ($i$). For enhver tolerancegrad $\epsilon > 0$ vil vi konstruere et $N > 0$ sådan at for alle $p\geq q \geq N$ har vi $d(y_q,y_p)<\epsilon$. Dette kan også skrives \hfill \break
\begin{align}
d(y_q,y_{q+k})<\epsilon,  \forall k \geq 0, \forall q \geq N
\end{align}
Hvis $k \geq 1$ antyder trekantsuligheden, at \hfill \break
\begin{align}
d(y_q,y_{q+k}) &\leq d(y_{q},y_{q+1})+d(y_{q+1},y_{q+k}) \notag\\
&\leq d(y_q,y_{q+1})+d(y_{q+1},y_{q+2})+d(y_{q+2},y_{q+k}) \notag\\
&\leq \sum\limits_{r=0}^{k-1} d(y_{q+r},y_{q+r+1})
\end{align}
For ethvert $n\geq 1$ har vi \hfill \break
$$d(y_n,y_{n+1})=d(F(y_{n-1}),F(y_n)) \leq Ld(y_{n-1},y_n) \leq \hdots \leq L^{n-1}d(y_1,y_2), \forall n \geq 1$$
Dermed er $d(y_{q+r},y_{q+r+1}) \leq L^{n-1}d(y_1,y_2)$ for alle $q \geq 1$ og $r\geq 0$. Dette og (2.4) giver, at \hfill \break
$$d(y_q,y_{q+k}) \leq L d(y_1,y_2)(1+\hdots+L^{k-1}) \leq \frac{L^{q-1}}{1-L}d(y_1,y_2), \forall k \geq 1$$
ved at bruge $(1-L)(1+L^2+\hdots+L^{k-1})=1+L^2+\hdots +L^{k-1}-L-L^2-\hdots -L^k= 1-L^k \leq 1$. \hfill \break
Da $0 \leq L < 1$ må $\lim_{q\rightarrow\infty} L^q=0$ og (2.3) må gælde. Det kan dermed konkluderes, at der eksisterer $y_\infty \in X$ sådan at \hfill \break
$$\lim_{n \to \infty} d(y_n,y_\infty)=0$$
Dermed må $\{y_n\}$ være Cauchyfølge i $M$. \hfill \break

Nu bevises ($ii$). For ethvert $n \geq 1$ gælder:
$$d(F(y_\infty),y_\infty)\leq d(F(y_\infty),F(y_n))+d(F(y_n),y_\infty)$$
Da $d(F(y_\infty),F(y_n))\rightarrow 0$ og $\lim_{n \to \infty} d(F(y_n),y_\infty) = \lim_{n \to \infty} d(y_{n+1},y_\infty)=0$, så må $d(F(y_\infty),y_\infty)=0$ og dermed $F(y_\infty)=y_\infty$. Så $y_\infty$ er unikt fixpunkt for $F$.
\end{proof}

\subsection{Bevis af Picard-Lindelöf sætningen}

\begin{definition}[Normeret vektorrum og Banachrum]
Hvis et vektorrum $V$ over et legeme $\mathbb{F}$ har normen af alle vektorer i $V$ defineret ved funktionen $||\cdot||:V\to \mathbb{R}_+$, kaldes det ordnede par $(V,||\cdot||)$ et normeret vektorrum.
Hvis alle Cauchyfølger af vektorer $\{\vec{v}_n\}$ i $V$ konvergerer mod en anden vektor $\vec{v}_\infty$ i $V$, kaldes det normerede vektorrum et fuldstændigt normeret vektorrum eller et Banachrum.
\end{definition}

\begin{proof}
Det var beviset.
\end{proof}

\subsection{Jacobi-matricer}

For at kunne beskrive ikke-lineære systemer omkring et ligevægtspunkt, indføres følgende

\begin{definition}[Jacobi-matrix]
Lad $$\textbf{f}: \mathbb{R}^n \to \mathbb{R}^m$$ være en funktion der afbilleder en vektor, $$\vec{x} \in \mathbb{R}^n$$ over i $$\textbf{f}(\vec{x}) \in \mathbb{R}^m \text{.}$$ \\ Da defineres en $m \times n$ matrix, $\textbf{J}$, på $\textbf{f}$ til at være den Jacobianske matrix til funktionen og skrives på formen:
$$\textbf{J}_f(\vec{x}) = \frac{d\textbf{f}}{d\vec{x}} =
\begin{bmatrix}
    \frac{\partial \textbf{f}_1}{\partial \vec{x_{1}}} & \frac{\partial \textbf{f}_1}{\partial \vec{x_{2}}} & \dots & \frac{\partial \textbf{f}_1}{\partial \vec{x_{n}}} \\
    \frac{\partial \textbf{f}_2}{\partial \vec{x_{1}}} & \frac{\partial \textbf{f}_2}{\partial \vec{x_{2}}} & \dots & \frac{\partial \textbf{f}_2}{\partial \vec{x_{n}}} \\
    \vdots & \vdots & \ddots & \vdots \\
    \frac{\partial \textbf{f}_m}{\partial \vec{x_{1}}} & \frac{\partial \textbf{f}_m}{\partial \vec{x_{2}}} & \dots & \frac{\partial \textbf{f}_m}{\partial \vec{x_{n}}}
\end{bmatrix},$$
eller komponentvist:
$$\textbf{J}_{ij} = \frac{\partial \textbf{f}_i}{\partial \vec{x}_j}$$

Den Jacobianske matrix er en lineær afbildning $\mathbb{R}^n \to \mathbb{R}^m$, der angiver en lineær approksimation af $\textbf{f}$ til et $\vec{x}$.
\end{definition}

\begin{definition}[]
Lad $\textbf{f}$ være en funktion og $a \in \mathbb{R}^n$ være et punkt. Er $\textbf{f}$ differentiabel i $a$, er dens afledte givet ved $\textbf{J}_f({a})$. I dette tilfælde siges den lineære afbildning givet den Jacobianske-matrix at være den bedste lineære approksimation af $\textbf{f}$ til $a$.

\hfill \break
Denne lineære approksimation skrives:

$$\textbf{f}(x)=\textbf{f}(a)+ \textbf{J}_f({a})(x-a)+ o(\left. \Vert \left. x-a \right. \Vert \right.),$$ 
for $x \to a$.

\end{definition}

\begin{definition}[Hyperbolsk ligevægtspunkt]
%asfAFASF
\end{definition}

\begin{Example}

$$\textbf{F(x)} =
\begin{bmatrix}
x_1 x_3^2 -e^{x_1 x_2} \\
x_2 - \frac{1}{x_3} \\
x_1 x_2 x_3
\end{bmatrix} \Rightarrow
J_{f}(x) =
\begin{bmatrix}
    x_3^2 - x_2 e^{x_1 x_2} & -x_1 e^{x_1 x_2} & 2x_1 x_3 \\
    0 & 1 & \frac{1}{x_3^2}\\
    x_2 x_3 & x_1 x_3 & x_1 x_2
\end{bmatrix}$$

\end{Example}



\chapter*{Grænseværdi}

\begin{definition}[Punktfølge]
En punktfølge består af uendeligt mange nummererede elementer i $\mathbb{R}^2$:

\begin{equation}
    x^1, x^2, \hdots, x^k, \hdots
\end{equation}
\end{definition}

\begin{definition}[Konvergent punktfølge]
En punktfølge $\{x^k\}^\infty_{k=1}$ i $\mathbb{R}^n$ siges at være konvergent, hvis der eksisterer et punkt $x \in \mathbb{R}^n$ således, at 

\begin{equation}
    \lim_{k \to \infty} ||x-x^k|| = 0.
\end{equation}

I så fald kaldes $x$ for følgens grænsepunkt, grænsevektor, eller grænseværdi, vi siger, at punktfølgen konvergerer mod $x$, og vi skriver

\begin{equation}
    x^k \to x \ for \ k \to \infty \ eller \ \lim_{k \to \infty} x^k = x. 
\end{equation}

En punktfølge, der ikke er konvergent, siges at være divergent. 

\end{definition}

\begin{definition}[]
En punktfølge $\{x^k\}^\infty_{k=1}$ siges at være begrænset, hvis der eksisterer et positivt reelt tal $M$, så $||x^k|| \leq M$ for alle $k\in \mathbb{N}$. 
\end{definition}

\begin{mytheo}{}
h
\begin{enumerate}[label=(\alph*)]
    \item En punktfølge kan højst have et grænsepunkt. 
    \item Hvis en punktfølge er konvergent, er den begrænset. 
    \item Lad $\{x^k\}^\infty_{k=1}$ være en punktfølge $x$ et punkt i $\mathbb{R}^n$. Så gælder
    
    \begin{equation}
        x^k \to x \ for \ k \to \infty \Leftrightarrow x^k-x \to 0 \ for \ k \to \infty.
    \end{equation}
    
    \item Lad $\{x^k\}^\infty_{k=1}$ være en punktfølge. Så gælder
    
    \begin{equation}
        x^k \to 0 \ for \ k \to \infty \Leftrightarrow ||x^k|| \to 0 \ for \ k \to \infty.
    \end{equation}
    
\end{enumerate}

\end{mytheo}


\begin{mytheo}{Regneregler for grænseværdier}{dd}

\begin{enumerate}[label=(\alph*)]
    \item Hvis punktfølgen $\{x^k\}^\infty_{k=1}$ er konvergent, så er talfølgen $\{||x^k||\}^\infty_{k=1}$ konvergent, og 
    \begin{equation}
        \lim_{k \to \infty} ||x^k|| = || \lim_{k \to \infty} x^k||.
    \end{equation}
    
    \item Hvis punktfølgerne $\{x^k\}^\infty_{k=1}$ og $\{y^k\}^\infty_{k=1}$ er konvergente, så er punktfølgerne \\
    $\{x^k \pm y^k\}^\infty_{k=1}$ konvergente, og 
    
    \begin{equation}
        \lim_{k \to \infty} (x^k \pm y^k) =  \lim_{k \to \infty} x^k \pm \lim_{k \to \infty} y^k
    \end{equation}
    
    \item Hvis $\{a_k\}^\infty_{k=1}$ er en konvergent talfølge og $\{x^k\}^\infty_{k=1}$ er en konvergent punktfølge, så er punktfølgen $\{a_kx^k\}^\infty_{k=1}$ konvergent, og 
    
    \begin{equation}
        \lim_{k \to \infty} a_k x^k =  \lim_{k \to \infty} a_k \lim_{k \to \infty} x^k
    \end{equation}
    
    \item Hvis punktfølgerne $\{x^k\}^\infty_{k=1}$ og $\{y^k\}^\infty_{k=1}$ er konvergente, så er talfølgen $\{\langle x^k,y^k\rangle\}^\infty_{k=1}$ konvergent, og der gælder 
    
    \begin{equation}
         \lim_{k \to \infty} \langle x^k,y^k\rangle =   \langle \lim_{k \to \infty} x^k, \lim_{k \to \infty} y^k \rangle.
    \end{equation}
    
\end{enumerate}

\end{mytheo}

%TO-DO:
%Definer at Jacobi-matricens determinant samt egenværdier angiver typen af ligevægtspunkt.
%Eventuelt definer Egenværdier og egenvektorer.

\section{Diagonaliserbare matricer}

\begin{mytheo}{Diagonalisering ved egenværdier og egenvektorer}{Diagegvd}
Hvis egenværdierne $\lambda_1, \lambda_2, \hdots, \lambda_n$, i en $n\times n$ matrix $A$, er reelle og forskellige, så vil enhver mængde af egenvektorer \{$\vec v_1, \vec v_2, \hdots, \vec v_n$\}, hørende til $\lambda_1, \lambda_2, \hdots, \lambda_n$, have følgende egenskaber:
\hfill \break
\begin{enumerate}
    \item \{$\vec v_1, \vec v_2, \hdots, \vec v_n$\} former en basis for $\mathbb{R}^n$
    \item Matricen $P = (\vec v_1, \vec v_2, \hdots, \vec v_n)$ er invertibel
    \item $P^{-1}AP = \textnormal{diag}[\lambda_1, \hdots, \lambda_n]$
\end{enumerate}

\end{mytheo}

\begin{koro}{}{}
Under hypoteserne givet i \ref{th:Diagegvd}, vil løsningen til det lineære system være givet ved funktion $y(t)$, som defineret ved (brødtekst ovenfor)
\end{koro}

\section{Operatorer}

\begin{definition}[Konvergens af lineære operatorer]
En følge af lineære operatorer $T_k \in L(\mathbb{R}^n)$ konvergerer mod en anden lineær operator $T \in L(\mathbb{R}^n)$, når $k \to \infty$
$$\lim_{k\to\infty} T_k = T,$$
Hvis der $\forall \varepsilon > 0 \exists N \in \mathbb{N}: k \geq N \to ||T - T_k|| < \varepsilon$
\end{definition}

\begin{lemma}{}{}
Lad $S,T \in L(\mathbb{R}^n)$ og $x \in \mathbb{R}^n$, så gælder følgende:
\begin{enumerate}
    \item $|T(y)| \leq ||T|| |y|$
    \item $||TS|| \leq ||T|| ||S||$
    \item $||T^k|| \leq ||T||^k$ for $k = 0,1,2, \hdots$
\end{enumerate}
\end{lemma}
\subsection{Den Fundamentale sætning for lineære systemer}

For et lineært systems koefficientmatrix, A, gælder:
\begin{lemma}{}{}
Lad $A$ være en kvadratisk matrix, så vil:
$$\frac{d}{dt}e^{At} = Ae^{At}$$
\end{lemma}

\begin{mytheo}{Den fundamentale sætning for lineære systemer}{}
Lad $A$ være en $n$ x $n$ matrix. Så vil der for et givet $y_0 \in \mathbb{R}^n$ eksistere en entydig løsning til det associerede IVP ved:
$$y(t) = e^{At}y_0$$
\end{mytheo}

Det kan altså konkluderes at Lotka-Volterra's model er et autonomt, ikke-koblet og ikke-lineært differentialligningssystem.

%Hvordan løser man et ikke-lineært system?
%Hvordan løser man et afkoblet system?
%Kan man muligvis smide 3.1.1 (den fundamentale sætning for lineære systemer) i bilaget?
%Første orden

Vi benytter Laplacetransformation, hvis vi har begyndelsesværdier, da problemet kan reduceres til algebra (Se sidst i afsnittet for laplacetransformen af Lotka-Volterra):
\begin{Example}
\textnormal{Lad os betragte følgende system af ODEs med begyndelsesværdier, $x(0)=1$ og $y(0)=0$.}
\begin{align}
    2x' + y' - y &= t\\
    x' + y' &= t^2
\end{align}
\textnormal{Vi tager laplacetransfomationen af alt}
\begin{align*}
    2(s X(s) - x(0)) + sY(s) - y(0) - Y(s) &= \frac{1}{s^2}\\
    s X(s) - x(0) + sY(s) - y(0) &= \frac{2}{s^3}\\
    2sX(s) + (s-1)Y(s) &= 2 + \frac{1}{s^2}\\
    sX(s) + sY(s) &= 1 + \frac{2}{s^3}
\end{align*}
\textnormal{Det er klart, at $X(s)$ nemt kan fjernes. Så det gør vi}
\begin{align*}
    2sX(s) + (s-1)Y(s) &= 2 + \frac{1}{s^2}\\
    -2(sX(s) + sY(s) &= 1 + \frac{2}{s^3})\\
    (s-1-2s)Y(s) &= \frac{1}{s^2} - \frac{4}{s^3}\\
    (-s-1)Y(s) &= \frac{s-4}{s^3}\\
    Y(s) &= \frac{4-s}{s^3(s + 1)}
\end{align*}
\textnormal{Vi omskriver nu til flere brøker (Ved ikke hvad partial fractions hedder på dansk)}
$$ \frac{4-2}{s^3(s+1)} = \frac{A}{s} + \frac{B}{s^2} + \frac{C}{s^3} + \frac{D}{s+1}$$
\textnormal{Der ganges igennem med $s^3(s+1)$}
$$ 4 - s = A s^2(s+1) + Bs(s+1) + C(s+1) + Ds^3$$
\textnormal{Kigger nu på ligningen og indser, at}
\begin{align*}
    A + D &= 0\\
    A + B &= 0\\
    B + C &= -1\\
    C &= 4
\end{align*}
\textnormal{Da får vi: $B = -5$, $A = 5$, $D = -5$}
$$Y(s) = \frac{5}{s} -\frac{5}{s^2} + \frac{4}{s^3} - \frac{5}{s+1}$$
\textnormal{Da anvender vi $\mathcal{L}^{-1}$}
$$ y(t) = 5 - 5t + 2t^2 - 5e^{-t}$$
\textnormal{vi isolerer nu X(s) og tager $\mathcal{L}^{-1}$ i den laplacetransformerede ligning 3.2}
\begin{align*}
    s(X)s - 1 + sY(s) &= \frac{2}{s^3}\\
    X(s) &= \frac{1}{s} + \frac{s}{s^4} - Y(s)\\
    x(t) &= 1 + \frac{1}{3}t^3 - (5 - 5t + 2t^2 - 5e^{-t})\\
    x(t) &= -4 + 5t-2t^2 + \frac{1}{3}t^3 + 5e^{-t}
\end{align*}

\end{Example}


For vores tilfælde kan vi se følgende:

\begin{equation*}
    \dfrac{db}{dt}(t) = (H-Ir(t)) b(t), 
\end{equation*}

\begin{equation*}
    \dfrac{dr}{dt}(t) = (Jb(t)-K) r(t),
\end{equation*}

$$b' - (H - Ir)b = 0 $$
$$r' - (Jb - K)r = 0 $$
Ved at tage laplacetransformen fås:
$$sB(s) - b(0) - (\frac{H}{s} - \frac{I}{s}R(s))B(s) = 0$$
$$sR(s) - r(0) - (\frac{J}{s}B(s) - \frac{K}{s})R(s) = 0$$
Så vi skal bruge nogle begyndelsesværdier ellers er vi fanget :O



\subsection{Inhomogene lineære andenordens ODE med konstante koefficienter}\label{ila} 
For fuldstændighedens skyld opsummeres de ubestemte koefficienters metode. Dette afsnit er baseret på \citep[s. 240-246]{JAB}.\hfill \break
En inhomogen lineær andenordens ODE med konstante koefficienter skrives på formen:

\begin{equation}
\label{inhomlinandord}
    a_2y''(t)+a_1y'(t)+a_0y(t)=f(t)
\end{equation} \hfill \break
Hvor $a_2,a_1,a_0\in \mathbb{R}$ er de konstante koefficienter og $a_2 \neq 0$. I dette afsnit vil de ubestemte koefficienters metode, som er en fremgangsmåde, hvorpå inhomogene andenordens ODE's kan løses, blive gennemgået.
Metoden går i alt sin enkelthed ud på, at hvis vi har givet en ligning på formen \eqref{inhomlinandord}, så giver vi et kvalificeret gæt på formen af $y_p(t)$. Har vi for eksempel givet en ligning på formen:
\begin{equation*}
    a_2y''(t)+a_1y'(t)+a_0y(t)=Ct^m
\end{equation*} \hfill \break
antyder formen $f(t)=Ct^m$, at $y_p(t)$ skal være et polynomium af $m'te$ orden:
\begin{equation*}
    y_p(t)=A_mt^m+\cdots +A_1t+A_0
\end{equation*} \hfill \break
Udregnes $y'_p(t)$ og $y''_p(t)$ kan disse sættes ind i \ref{inhomlinandord}, hvor de ubekendte koefficienter til hver potens af t i $a_2y''(t)+a_1y'(t)+a_0y(t)$ matches med de tilsvarende i $f(t)$. 
Hvis $f(t)=Ce^{at}$ 'gætter' vi på en løsning på formen $Ae^{at}$
\begin{Example}\hfill \break
\textnormal{For at finde en partikulær løsning til:}\hfill \break
$$3y''+2y'+4y=20e^{4t}$$ \hfill \break
\textnormal{gætter vi på at $y_p(t)=Ae^{4t}$, dermed bliver $y'=4Ae^{4t}$ og $y''=16Ae^{4t}$ og beholder dermed den samme eksponentielle form. Vi får altså:} \hfill \break
$$3y_p''+2y_p'+y_p=3(16Ae^{4t})+2(4Ae^{4t})+4Ae^{4t}=60Ae^ {4t}=20e^{4t}$$ \hfill \break
\textnormal{Hvilket giver $A=\frac{1}{3}$, og dermed} \hfill \break
$$y_p(t)=\frac{e^{4t}}{3}$$ \textnormal{som løsning.}

\end{Example}
Generelt gælder det at formen på $f(t)$ antyder formen på den partikulære løsning, som det ses i nedenstående tabel. 
\begin{table}[H]
    \centering
    \begin{tabular}{|l|l|}
    \hline
       $f(t)$  & $y_p(t)$ \\ \hline
        $Ct^m$ & $A_mt^m+\cdots +A_1t+A_0$ \\ \hline
        $Ce^{at}$ & $Ae^{at}$ \\ \hline
        $C sin(at) , C cos(at)$ & $ A cos(at) + B sin(at)$ \\ \hline
        $Ct^me^t$ & $(A_mt^m+ \cdots +A_1t+A_0)e^t$ \\ \hline
        \end{tabular}
\end{table}
I tilfælde af at $f(t)$ er en sum af flere led, kan $f(t)$ deles op i disse led, og ligning \ref{inhomlinandord} løses for hvert led, hvorefter disse løsninger adderes, hvilket giver en løsning til $f(t)$. \\
Hvis $f(t)$ er et produkt af flere funktioner, vil formen på $y_p(t)$ typisk kunne skrives som et produkt af de tilhørende 'gæt' for hver funktion.
Der er dog tilfælde hvor denne fremgangsmåde ikke kan bruges.

\begin{Example}\hfill \break
\textnormal{Betragt ligningen} $$y'' -3y'-4y=5e^{4t}$$ \textnormal{Det antages at en løsning er på formen} $$y_p(t)=Ae^{4t}$$ \textnormal{Udregnes $y'_p(t)$ og $y''_p(t)$ og sættes ind får vi dog:}
$$y'' -3y'-4y=16Ae^{4t}-12Ae^{4t}-4Ae^{4t}=0  \neq 5e^{4t}$$
\textnormal{Dette skyldes at $e^{4t}$ er en løsning til den tilhørende homogene ligning $y'' -3y'-4y=0$, og enhver konstant multiplikation vil derfor give 0 på højresiden af ligningen. For at løse dette problem er det nok at gange det sædvanlige 'gæt' med faktoren $t$.}
\end{Example}

\hfill \break
\begin{prop}{De ubestemte koefficienters metode}
EEn partikulær løsning til differentialligningen $a_2y''(t)+a_1y'(t)+a_0y(t)=Ct^me^{rt}$ kan skrives på formen: 
$$y_p(t)= t^s(A_nt^n+A_{n-1}t^{n-1}  \cdots +A_1t+A_0)e^{rt}$$ 
hvor 
\begin{enumerate}
    \item $s=0$ hvis $r$ ikke er en rod i den tilhørende karakteristiske ligning
    \item $s=1$ hvis $r$ er en simpel rod i den tilhørende karakteristiske ligning
    \item $s=2$ hvis $r$ er en dobbelt rod i den tilhørende karakteristiske ligning
\end{enumerate}
\end{prop}
\hfill \break

\begin{proof}\\
Lad en ligning på formen:
\begin{equation*}
a_2y''(t)+a_1y'(t)+a_0y(t)=Ct^me^{rt}
\end{equation*}
være givet. Vi gætter her på at løsningen er på formen:
\begin{equation*}
y_p(t)= (A_nt^n+A_{n-1}t^{n-1} + \hdots +A_1t+A_0)e^{rt}
\end{equation*}
Hvor potensen $n$ skal bestemmes så den matcher $m$.
Ledende af højeste orden i $y_p(t)$,$y'_p(t)$ og $y''_p(t)$ ser ud som følger:
\begin{align*}
     y_p(t) &=e^{rt}(A_nt^n+A_{n-1}t^{n-1}+A_{n-2}t^{n-2}+\hdots + A_1t+A_0) \\ \break
    y'_p(t) &=e^{rt}(A_nrt^n+A_nnt^{n-1}+A_{n-1}rt^{n-1}+A_{n-1}(n-1)t^{n-2}+A_{n-2}rt^{n-2}+\hdots + A_1rt+A_1+A_0r) \\
    y''_p(t) &=e^{rt}(A_nr^2t^n+2A_nnrt^{n-1}+A_nn(n-1)t^{n-2}+A_{n-1}r^2t^{n-1}+2A_{n-1}r(n-1)t^{n-2} \\ 
    &+A_{n-2}r^2t^{n-2}+\hdots + A_1r^2t+2A_1r+A_0r^2)
\end{align*}
Når dette sættes ind i $a_2y''(t)+a_1y'(t)+a_0y(t)$ fås:
\begin{align*}
 a_2y''(t)+ & a_1y'(t)+a_0y(t) \\ 
 & = A_n(a_2r^2+a_1r+a_0)t^ne^{rt} + (A_nn(2a_2r+a_1) +A_{n-1}(a_2r^2+a_1r+a_0))t^{n-1}e^{rt} \\ 
 & +(A_nn(n-1)a_2+A_{n-1}(n-1)(2a_2r+a_1)\\
 &+A_{n-2}(a_2r^2+a_1r+a_0))t^{n-2}e^{rt}+ \hdots + (led\ af\ lavere\ orden)
\end{align*}
Da den karakteristiske ligning for $a_2y''(t)+a_1y'(t)+a_0y(t)$ er: \\
$$a_2r^2+a_1r+a_0=0$$ \\
og hvis $r$ er en dobbelt rod\\
$$2a_2r+a_1=0$$\\ 
Der er altså tre mulige scenarier: \\

\textbf{1.} \\
Hvis $r$ er ikke en rod i den karakteristiske ligning, og dermed vil leddet med den største potens være $A_n(a_2r^2+a_1r+a_0)t^ne^rt$. For at matche potensen i $f(t)=Ct^me^{rt}$, må $n=m$ og vi får: \\
\begin{equation*}
y_p(t)= (A_mt^m+A_{m-1}t^{m-1}  \cdots +A_1t+A_0)e^rt
\end{equation*} \\

\textbf{2.} \\
Hvis $r$ er en simpel rod i den karakteristiske ligning hvilket medfører at $a_2r^2+a_1r+a_0=0$ bliver leddet med den højeste potens $(A_nn(2a_2r+a_1)$. For at matche potensen i $f(t)=Ct^me^{rt}$, må $n=m+1$ og vi får: \\
\begin{equation*}
y_p(t)= (A_{m+1}t^{m+1}+A_{m}t^{m}  \cdots +A_1t+A_0)e^{rt}
\end{equation*} \\
Men da r  er en simpel rod i den karakteristiske ligning, vil $A_0e^{rt}$ være  en løsning til den karakteristiske ligning, og derfor kan dette led fjernes og $t$ sættes uden for en parentes. \\
\begin{align*}
y_p(t)&= (A_{m+1}t^{m+1}+A_{m}t^{m}  \cdots+ A_2t +A_1)e^{rt} \\
      &= t(A_{m}t^{m}+A_{m-1}t^{m-1}  \cdots +A_1t+A_0)e^{rt}
\end{align*} \\

\textbf{3.}\\
Hvis $r$ er en dobbelt rod i den karakteristiske ligning, hvilket medfører, at $2a_2r+a_1=0$, så er leddet med den højeste potens $A_nn(n-1)a_2t^{n-2}e^rt$. For at matche potensen i $f(t)=Ct^me^{rt}$, må $n=m+2$ og vi får: \\

\begin{equation*}
y_p(t)= (A_{m+2}t^{m+2}+A_{m+2}t^{m+2}  \cdots +A_1t+A_0)e^{rt}
\end{equation*} \\
Som før fjernes $(A_1t+A_0)e^{rt}$ da $A_1te^{rt}$ og $A_0e^{rt}$ er løsninger til den tilhørende homogene ligning, og $t^2$ sættes uden for parentes: 

\begin{align*}
y_p(t)&= (A_{m+2}t^{m+2}+A_{m+1}t^{m+1}  \cdots+ A_2t^2+A_1t+A_0)e^{rt} \\
      &= t^2(A_{m}t^{m}+A_{m-1}t^{m-1}  \cdots +A_1t+A_0)e^{rt}
\end{align*} \\

\end{proof}

\subsection{Lineære differentialligninger}
Følgende er med til at skabe en forståelse for løsninger til lineære differentialligninger og afdækkes, da de første systemer af differentialligninger, der introduceres i projektet, er lineære.
\hfill \break

En førsteordens OLDE er en ligning på formen: \\ 
$$a_{1}(t) \frac{dy}{dt} + a_{0}(t)y = f(t)$$ Hvor $t$ er den uafhængige variabel. \hfill \break

Vi betragter nu to tilfælde for koefficienten $a_0$, hvor $a_0(t) = 0$ og $a_0(t) = a_1'(t)$. Hvis $a_0(t) = 0$ reduceres ligningen til $a_1(t)\frac{dy}{dt} = f(t)$.  
\begin{Example}\hfill \break
\textnormal{Antag at vi har en førsteordens OLDE, som følger, hvor $a_0(t) = 0$.}\\
\hfill \break
\centerline{$0y - 6t = -(3t^2) \frac{dy}{dt}$ $\Rightarrow$ $3t^2 \frac{dy}{dt} = 6t$}
\hfill \break
\centerline{$y'(t) = \frac{6t}{3t^2}$ $\Rightarrow$ $y(t) = \int \frac{6t}{3t^2}dt$}
\hfill \break
\textnormal{Løsningen kan derfor findes ved simpel isolering af variable og integration.}
\end{Example}

For det andet tilfælde får vi af produktreglen at $$a_1(t)y'(t) + a_0(t)y(t) = a_1(t)y'(t) + a_1'(t)y(t) = \frac{d}{dt}(a_1(t)y(t))$$Da har vi $$\frac{d}{dt}(a_1(t)y(t)) = f(t)$$ hvoraf løsningen let kan findes.

\begin{Example} \hfill \break
\textnormal{Antag at vi har en førsteordens OLDE, som følger, hvor $a_0(t) = a_1'(t)$} \\
\hfill \break
$$t^2 = \frac{1}{t}y'(t) -\frac{1}{t^2}y(t) = \frac{d}{dt}(\frac{1}{t}y(t))\Leftrightarrow$$
$$\frac{d}{dt}(\frac{1}{t}y(t)) = t^2\Leftrightarrow$$
$$\frac{1}{t}y(t) = \int t^2dt \Leftrightarrow$$ $$ y(t) = t \int t^2dt$$
\hfill \break
\textnormal{Dermed kan vi igen isolere og integrere for at finde løsningen.}
\end{Example}

Der eksisterer mange tilfælde, hvor OLDE ikke umiddelbart kan reduceres til en af de to ovenstående former, hvorfor det ses nødvendigt at finde en anden metode til løsning af disse. Derfor omskriver vi OLDE, som vist i definition \ref{OLDE}: $${y'(t) + \frac{a_0(t)}{a_1(t)}y(t) = \frac{b(t)}{a_1(t)}} \Leftrightarrow$$
$$y'(t) + P(t)y(t) = Q(t)$$ hvor $y'(t) + P(t)y(t) = Q(t)$ er den normerede ligning. Derudover vil vi anvende en integrationsfaktor.

\begin{definition}[Integrationsfaktor]\label{IntFak}
En funktion $\mu (t)$, der ved multiplikation ændrer en lineær normeret differentialligning til en ligning på formen: $$\frac{d}{dt}(\mu (t)y(t)) = \mu (t)Q(t)$$ kaldes en integrationsfaktor.  
\end{definition}

Denne integrationsfaktor kan benyttes til at finde en generel løsning til en OLDE, og det er illustreret ved nedenstående sætning.

\begin{mytheo}{Løsning til OLDE}{}
For enhver OLDE, der kan skrives på formen:
\begin{equation}\label{LOLDE}
\frac{dy}{dt}+P(t)y=Q(t)
\end{equation}
eksisterer der en integrationsfaktor $\mu(t)$, hvorom der gælder, at: $$y(t)=\frac{1}{\mu (t)}(\int \mu (t)Q(t)dt+K)$$ er den generelle løsning til OLDE.
\end{mytheo}

\begin{proof}
Vi anvender en integrationsfaktor $\mu(t)$ ved at gange denne til den normerede ligning.
\begin{equation}\label{Intfaktor}
\mu (t)y'(t) + \mu (t)P(t)y(t) = \mu (t)Q(t) 
\end{equation}
Per definition \ref{IntFak} må $\mu ' = \mu P$ og da kan vi udlede af \ref{th:LSD} at $\frac{1}{\mu}d\mu = P(t)dt$.
Ligning \eqref{Intfaktor} omskrives til $$\frac{d}{dt}(\mu (t)y(t)) = \mu (t)Q(t)$$ 
med $\mu (t)$ som den ovenstående, kan vi finde en generel løsning til (\ref{LOLDE}) ved integration på begge sider og derefter isolere $y(t)$:
\begin{equation}
y(t)=\frac{1}{\mu (t)}(\int \mu (t)Q(t)dt+K)
\end{equation}
\end{proof}

\begin{Example}
\textnormal{Antag at vi har en førsteordens OLDE:}
$$2ty'(t) + 4y(t) = 6t^2$$
\textnormal{Hvis vi dividerer igennem med 2t fås:}
$$y'(t) + \frac{2}{t}y(t) = 3t$$
$$\int P(t)dt = \int \frac{2}{t}dt = 2ln(t)$$
\textnormal{Da har vi integrationsfaktoren,} $\mu = e^{2ln|t|} = e^{ln(t^2)} = t^2$. \textnormal{Nu ganges $\mu$ på den normerede ligning:}
$$t^2 \frac{dy}{dt} + 2ty = 3t^3\Leftrightarrow$$
$$\frac{d}{dt}(t^2y) = 3t^3 \Leftrightarrow  t^2y = \int 3t^3dt = \frac{3}{4}t^4 + K$$
\textnormal{Da får vi:} 
$$y = \frac{3}{4}t^4t^{-2} + Kt^{-2} = \frac{3}{4}t^2 + K \frac{1}{t^2}$$
\end{Example}

\begin{Example}\textbf{Pendul med friktion}\label{friktion}
\hfill \break
\textnormal{Betragt differentialligningen for et pendul med friktion}\hfill \break
\begin{align*}
y''+cy'+a\cdot sin(y)=0, \ c,a > 0
\end{align*}

\textnormal{Denne differentialligning af anden orden kan omskrives til et system af første orden ved substitution:}
\begin{align*}
y_1&=y \\
y_2&=y'
\end{align*}
\textnormal{Dermed:}
\begin{align*}
y_1'&=y_2 \\
y_2'&=-cy_2-a\cdot sin(y_1)
\end{align*}
\textnormal{Den mekaniske energi i systemet er givet ved:}
\begin{align}
E(y_1,y_2)=\frac{1}{2}y_2^2+a(1-cos(y_1))
\end{align}
\textnormal{Man kunne forestille sig, at den mekaniske energi ikke er konstant i dette system på grund af friktion. Dette bør kontrolleres ved at bestemme differentialkvotienten til den mekaniske energi i systemet. Ved kædereglen kan denne bestemmes:}
\begin{align*}
\frac{d}{dt}E(y_1(t),y_2(t))&=\frac{\partial E}{\partial y_1}(y_1,y_2)y_1'(t)+\frac{\partial E}{\partial y_2}(y_1,y_2)y_2'(t)\\
&=a\cdot sin(y_1)y_2+y_2(-cy_2-a\cdot sin(y_1))\\
&=-cy_2^2\\
&\leq 0
\end{align*}
\textnormal{Den mekaniske energi er dermed en aftagende funktion med hensyn til $t$. Når en løsningskurve skærer en niveaukurve $E(y_1,y_2)=k$, har denne kurve retning mod aftagende $k$ og dermed imod origo, da $E(0,0)=0$. Dermed bør origo være et asymptotisk stabilt ligevægtspunkt. Dette vil også vise sig at være tilfældet senere.}
\end{Example}
