\chapter{Ekstra-stuff}

\subsection{Mikkel}

\subsection{Lipschitz-kontinuitet}
Lipcshitz-kontinuitet er første skridt i at bevise eksistens- og entydighedssætningen for løsninger til differentialligninger?
\begin{definition}[Lipschitz-kontinuitet]
Givet en funktion, $f: (t,\vec{x})\in R \subseteq \mathbb{R}^{m+1} \rightarrow f(t,\vec{x}) \in \mathbb{R}^m$, kaldes Lipschitz-kontinuert, hvis $\exists L \geq 0$, så:
$$|f(t,\vec{x_1})-f(t,\vec{x_2})|\leq L|\vec{x_1}-\vec{x_2}|$$
\end{definition}

\begin{Example}\textbf{Lipschitz-kontinuitet}\hfill \break
\textnormal{Hvis vi lader $\mathbb{R}^2$ være det réelle plan og betragter funktionen $f(t,x)=t^5+4x$, så kan vi undersøge om $f$ er Lipschitz-kontinuert på $\mathbb{R}^2$:} \hfill \break
$$|f(t,x_1)-f(t,x_2)|=|(t^5+4x_1)-(t^5+4x_2)|=|4x_1-4x_2|=4|x_1-x_2|$$ \hfill \break
\textnormal{Det ses nu, at funktionen opfylder en Lipschitz betingelse på $\mathbb{R}^2$ med $L=4$.}
\end{Example}
\begin{lemma}{}{}
Lad $D$ være et lukket rektangel $R=(a,b)\times(c,d)$ eller en uendelig strimmel $S=(a,b)\times(-\infty,\infty)$ på planet $\mathbb{R}^2$. $f$ er Lipschitz-kontinuert på $R$ eller $S$ hvis:

\begin{enumerate}
    \item $f:D \rightarrow \mathbb{R}$
    \item $\frac{\partial f}{\partial y}$ eksisterer og er kontinuert
    \item Der eksisterer en konstant $K\geq 0$ sådan, at $|\frac{\partial f}{\partial y}(t,y)|\leq K$
\end{enumerate}

Dermed er $K=L$
\end{lemma}
\begin{proof} \hfill \break
Ved brug af fundamental calculus gælder, at for alle $(t,y_1),(t,y_2) \in D$ er
\begin{align}
 f(t,y_1)-f(t,y_2)=\int_{y_2}^{y_1}\frac{\partial f}{\partial y} dy \notag
\end{align}
og derved
\begin{align}
 |f(t,y_1)-f(t,y_2)|&=|\int_{y_2}^{y_1}\frac{\partial f}{\partial y} dy| \notag \\
&\leq |\int_{y_2}^{y_1}|\frac{\partial f}{\partial y}| dy| \notag\\
&\leq |\int_{y_2}^{y_1}K| \notag \\
&=K|y_1-y_2| \notag
\end{align}
hvilket giver $K=L$ ifølge definitionen af Lipschitz-kontinuitet. \qedhere
\end{proof}
\hfill \break

\begin{Example}\hfill \break
\textnormal{Betragt funktionen $f(t,y)=t^5+\tan^{-1}(y)$. Hvis vi benytter Lemma 2.3.1 får vi:} \hfill \break
\begin{align}
\frac{\partial f}{\partial y}=\frac{1}{1+y^2} \notag
\end{align}
\textnormal{Det ses hurtigt, at brøken $\frac{1}{1+y^2}$ altid vil være mindre end eller lig med 1 for alle $(t,y)\in \mathbb{R}^2$, altså:}
\begin{align}
 \frac{1}{1+y^2}\leq 1, \forall(t,y) \in \mathbb{R}^2 \notag
\end{align}
\textnormal{Dermed ses, at $L=1$, og $f$ er Lipschitz-kontinuert på $\mathbb{R}^2$.}
\end{Example}

\subsection{Banachs fixpunktssætning}

\begin{definition}[Metrisk rum]
Et metrisk rum er et ordnet par $(M,d)$, hvor $M$ er en mængde og $d$ er en metrik $d:M\times M \rightarrow \mathbb{R}_+$, $d(x,y)=|x-y|$, hvorom der gælder for ethvert $x,y,z \in M$: 
\begin{enumerate}
    \item $d(x,y)=0 \Leftrightarrow x=y$ 
    \item $d(x,y)=d(y,x)$
    \item $d(x,z)\leq d(x,y)+d(y,z)$
\end{enumerate}
$d$ kaldes også for afstandsfunktionen.
\end{definition}

\begin{definition}[Cauchyfølger i Metriske rum]
En talfølge $\{x_n\}$ i et metrisk rum $(M,d)$ siges at være Cauchyfølge, hvis der for enhver positiv tolerencegrad $\varepsilon >0$ eksisterer $N \in \mathbb{N}$, så der for alle $m,n \geq N$, $m,n \in \mathbb{N}$ gælder, at: 
$$|x_n-x_m|<\varepsilon$$
\end{definition}

\begin{definition}[Fuldstændigt metrisk rum]
Et metrisk rum $(M,d)$ siges at være fuldstændigt, hvis alle Cauchyfølger i $M$ har en grænse, der også er i $M$. Altså hvis alle Cauchyfølger i $M$ konvergerer i $M$.
\end{definition}

\begin{definition}[Kontraktion og Fixpunkt]
Lad $(M,d)$ være et metrisk rum. En afbildning $F:M\rightarrow M$ kaldes en kontraktion, hvis der eksisterer en Lipschitz-konstant $0 \leq L < 1$, sådan at:
$$|F(x_1)-F(x_2)|\leq L|x_1-x_2|$$
Et punkt $x \in M$ er et fixpunkt for $F$, hvis $F(x)=x$ 
\end{definition}

\begin{mytheo}{Banachs fixpunktssætning}
LLad $(X,d)$ være et fuldstændigt metrisk rum og $F:X\rightarrow X$ være en kontraktion. Så har $F$ et unikt fixpunkt.
\end{mytheo}

\begin{proof}
Først viser vi unikheden af fixpunktet. Antag at der eksisterer $a,b \in M$, så $F(a)=a$ og $F(b)=b$, så antyder Definition 2.12, at \hfill \break
$$0\leq d(a,b)=d(F(a),F(b))\leq Ld(a,b), (1-L)|y_1-y_2|\leq 0$$
hvilket betyder at $d(a,b)=0$ og at $a=b$ \hfill \break
\hfill \break
Vi konstruerer nu sådan et fixpunkt. Betragt talfølgen ${y_n}$, hvor $y_1$ er arbitrær og $y_n:=F(y_{n-1})$ for ethvert $n\geq 2$. Vi ønsker nu at vise to ting:\hfill \break
 ($i$) Talfølgen er Cauchyfølge i M og dermed konvergerer mod et $y_\infty$, da vi antog at $M$ var fuldstændigt metrisk rum.\hfill  \break
($ii$) $y_\infty$ er et fixpunkt for $F$. \hfill \break
Vi starter med ($i$). For enhver tolerancegrad $\epsilon > 0$ vil vi konstruere et $N > 0$ sådan at for alle $p\geq q \geq N$ har vi $d(y_q,y_p)<\epsilon$. Dette kan også skrives \hfill \break
\begin{align}
d(y_q,y_{q+k})<\epsilon,  \forall k \geq 0, \forall q \geq N
\end{align}
Hvis $k \geq 1$ antyder trekantsuligheden, at \hfill \break
\begin{align}
d(y_q,y_{q+k}) &\leq d(y_{q},y_{q+1})+d(y_{q+1},y_{q+k}) \notag\\
&\leq d(y_q,y_{q+1})+d(y_{q+1},y_{q+2})+d(y_{q+2},y_{q+k}) \notag\\
&\leq \sum\limits_{r=0}^{k-1} d(y_{q+r},y_{q+r+1})
\end{align}
For ethvert $n\geq 1$ har vi \hfill \break
$$d(y_n,y_{n+1})=d(F(y_{n-1}),F(y_n)) \leq Ld(y_{n-1},y_n) \leq \hdots \leq L^{n-1}d(y_1,y_2), \forall n \geq 1$$
Dermed er $d(y_{q+r},y_{q+r+1}) \leq L^{n-1}d(y_1,y_2)$ for alle $q \geq 1$ og $r\geq 0$. Dette og (2.4) giver, at \hfill \break
$$d(y_q,y_{q+k}) \leq L d(y_1,y_2)(1+\hdots+L^{k-1}) \leq \frac{L^{q-1}}{1-L}d(y_1,y_2), \forall k \geq 1$$
ved at bruge $(1-L)(1+L^2+\hdots+L^{k-1})=1+L^2+\hdots +L^{k-1}-L-L^2-\hdots -L^k= 1-L^k \leq 1$. \hfill \break
Da $0 \leq L < 1$ må $\lim_{q\rightarrow\infty} L^q=0$ og (2.3) må gælde. Det kan dermed konkluderes, at der eksisterer $y_\infty \in X$ sådan at \hfill \break
$$\lim_{n \to \infty} d(y_n,y_\infty)=0$$
Dermed må $\{y_n\}$ være Cauchyfølge i $M$. \hfill \break

Nu bevises ($ii$). For ethvert $n \geq 1$ gælder:
$$d(F(y_\infty),y_\infty)\leq d(F(y_\infty),F(y_n))+d(F(y_n),y_\infty)$$
Da $d(F(y_\infty),F(y_n))\rightarrow 0$ og $\lim_{n \to \infty} d(F(y_n),y_\infty) = \lim_{n \to \infty} d(y_{n+1},y_\infty)=0$, så må $d(F(y_\infty),y_\infty)=0$ og dermed $F(y_\infty)=y_\infty$. Så $y_\infty$ er unikt fixpunkt for $F$.
\end{proof}

\subsection{Bevis af Picard-Lindelöf sætningen}

\begin{definition}[Normeret vektorrum og Banachrum]
Hvis et vektorrum $V$ over et legeme $\mathbb{F}$ har normen af alle vektorer i $V$ defineret ved funktionen $||\cdot||:V\to \mathbb{R}_+$, kaldes det ordnede par $(V,||\cdot||)$ et normeret vektorrum.
Hvis alle Cauchyfølger af vektorer $\{\vec{v}_n\}$ i $V$ konvergerer mod en anden vektor $\vec{v}_\infty$ i $V$, kaldes det normerede vektorrum et fuldstændigt normeret vektorrum eller et Banachrum.
\end{definition}

\begin{proof}
Det var beviset.
\end{proof}

\subsection{Jacobi-matricer}

For at kunne beskrive ikke-lineære systemer omkring et ligevægtspunkt, indføres følgende

\begin{definition}[Jacobi-matrix]
Lad $$\textbf{f}: \mathbb{R}^n \to \mathbb{R}^m$$ være en funktion der afbilleder en vektor, $$\vec{x} \in \mathbb{R}^n$$ over i $$\textbf{f}(\vec{x}) \in \mathbb{R}^m \text{.}$$ \\ Da defineres en $m \times n$ matrix, $\textbf{J}$, på $\textbf{f}$ til at være den Jacobianske matrix til funktionen og skrives på formen:
$$\textbf{J}_f(\vec{x}) = \frac{d\textbf{f}}{d\vec{x}} =
\begin{bmatrix}
    \frac{\partial \textbf{f}_1}{\partial \vec{x_{1}}} & \frac{\partial \textbf{f}_1}{\partial \vec{x_{2}}} & \dots & \frac{\partial \textbf{f}_1}{\partial \vec{x_{n}}} \\
    \frac{\partial \textbf{f}_2}{\partial \vec{x_{1}}} & \frac{\partial \textbf{f}_2}{\partial \vec{x_{2}}} & \dots & \frac{\partial \textbf{f}_2}{\partial \vec{x_{n}}} \\
    \vdots & \vdots & \ddots & \vdots \\
    \frac{\partial \textbf{f}_m}{\partial \vec{x_{1}}} & \frac{\partial \textbf{f}_m}{\partial \vec{x_{2}}} & \dots & \frac{\partial \textbf{f}_m}{\partial \vec{x_{n}}}
\end{bmatrix},$$
eller komponentvist:
$$\textbf{J}_{ij} = \frac{\partial \textbf{f}_i}{\partial \vec{x}_j}$$

Den Jacobianske matrix er en lineær afbildning $\mathbb{R}^n \to \mathbb{R}^m$, der angiver en lineær approksimation af $\textbf{f}$ til et $\vec{x}$.
\end{definition}

\begin{definition}[]
Lad $\textbf{f}$ være en funktion og $a \in \mathbb{R}^n$ være et punkt. Er $\textbf{f}$ differentiabel i $a$, er dens afledte givet ved $\textbf{J}_f({a})$. I dette tilfælde siges den lineære afbildning givet den Jacobianske-matrix at være den bedste lineære approksimation af $\textbf{f}$ til $a$.

\hfill \break
Denne lineære approksimation skrives:

$$\textbf{f}(x)=\textbf{f}(a)+ \textbf{J}_f({a})(x-a)+ o(\left. \Vert \left. x-a \right. \Vert \right.),$$ 
for $x \to a$.

\end{definition}

\begin{definition}[Hyperbolsk ligevægtspunkt]
%asfAFASF
\end{definition}

\begin{Example}

$$\textbf{F(x)} =
\begin{bmatrix}
x_1 x_3^2 -e^{x_1 x_2} \\
x_2 - \frac{1}{x_3} \\
x_1 x_2 x_3
\end{bmatrix} \Rightarrow
J_{f}(x) =
\begin{bmatrix}
    x_3^2 - x_2 e^{x_1 x_2} & -x_1 e^{x_1 x_2} & 2x_1 x_3 \\
    0 & 1 & \frac{1}{x_3^2}\\
    x_2 x_3 & x_1 x_3 & x_1 x_2
\end{bmatrix}$$

\end{Example}



\chapter*{Grænseværdi}

\begin{definition}[Punktfølge]
En punktfølge består af uendeligt mange nummererede elementer i $\mathbb{R}^2$:

\begin{equation}
    x^1, x^2, \hdots, x^k, \hdots
\end{equation}
\end{definition}

\begin{definition}[Konvergent punktfølge]
En punktfølge $\{x^k\}^\infty_{k=1}$ i $\mathbb{R}^n$ siges at være konvergent, hvis der eksisterer et punkt $x \in \mathbb{R}^n$ således, at 

\begin{equation}
    \lim_{k \to \infty} ||x-x^k|| = 0.
\end{equation}

I så fald kaldes $x$ for følgens grænsepunkt, grænsevektor, eller grænseværdi, vi siger, at punktfølgen konvergerer mod $x$, og vi skriver

\begin{equation}
    x^k \to x \ for \ k \to \infty \ eller \ \lim_{k \to \infty} x^k = x. 
\end{equation}

En punktfølge, der ikke er konvergent, siges at være divergent. 

\end{definition}

\begin{definition}[]
En punktfølge $\{x^k\}^\infty_{k=1}$ siges at være begrænset, hvis der eksisterer et positivt reelt tal $M$, så $||x^k|| \leq M$ for alle $k\in \mathbb{N}$. 
\end{definition}

\begin{mytheo}{}
h
\begin{enumerate}[label=(\alph*)]
    \item En punktfølge kan højst have et grænsepunkt. 
    \item Hvis en punktfølge er konvergent, er den begrænset. 
    \item Lad $\{x^k\}^\infty_{k=1}$ være en punktfølge $x$ et punkt i $\mathbb{R}^n$. Så gælder
    
    \begin{equation}
        x^k \to x \ for \ k \to \infty \Leftrightarrow x^k-x \to 0 \ for \ k \to \infty.
    \end{equation}
    
    \item Lad $\{x^k\}^\infty_{k=1}$ være en punktfølge. Så gælder
    
    \begin{equation}
        x^k \to 0 \ for \ k \to \infty \Leftrightarrow ||x^k|| \to 0 \ for \ k \to \infty.
    \end{equation}
    
\end{enumerate}

\end{mytheo}


\begin{mytheo}{Regneregler for grænseværdier}{dd}

\begin{enumerate}[label=(\alph*)]
    \item Hvis punktfølgen $\{x^k\}^\infty_{k=1}$ er konvergent, så er talfølgen $\{||x^k||\}^\infty_{k=1}$ konvergent, og 
    \begin{equation}
        \lim_{k \to \infty} ||x^k|| = || \lim_{k \to \infty} x^k||.
    \end{equation}
    
    \item Hvis punktfølgerne $\{x^k\}^\infty_{k=1}$ og $\{y^k\}^\infty_{k=1}$ er konvergente, så er punktfølgerne \\
    $\{x^k \pm y^k\}^\infty_{k=1}$ konvergente, og 
    
    \begin{equation}
        \lim_{k \to \infty} (x^k \pm y^k) =  \lim_{k \to \infty} x^k \pm \lim_{k \to \infty} y^k
    \end{equation}
    
    \item Hvis $\{a_k\}^\infty_{k=1}$ er en konvergent talfølge og $\{x^k\}^\infty_{k=1}$ er en konvergent punktfølge, så er punktfølgen $\{a_kx^k\}^\infty_{k=1}$ konvergent, og 
    
    \begin{equation}
        \lim_{k \to \infty} a_k x^k =  \lim_{k \to \infty} a_k \lim_{k \to \infty} x^k
    \end{equation}
    
    \item Hvis punktfølgerne $\{x^k\}^\infty_{k=1}$ og $\{y^k\}^\infty_{k=1}$ er konvergente, så er talfølgen $\{\langle x^k,y^k\rangle\}^\infty_{k=1}$ konvergent, og der gælder 
    
    \begin{equation}
         \lim_{k \to \infty} \langle x^k,y^k\rangle =   \langle \lim_{k \to \infty} x^k, \lim_{k \to \infty} y^k \rangle.
    \end{equation}
    
\end{enumerate}

\end{mytheo}

%TO-DO:
%Definer at Jacobi-matricens determinant samt egenværdier angiver typen af ligevægtspunkt.
%Eventuelt definer Egenværdier og egenvektorer.

\section{Diagonaliserbare matricer}

\begin{mytheo}{Diagonalisering ved egenværdier og egenvektorer}{Diagegvd}
Hvis egenværdierne $\lambda_1, \lambda_2, \hdots, \lambda_n$, i en $n\times n$ matrix $A$, er reelle og forskellige, så vil enhver mængde af egenvektorer \{$\vec v_1, \vec v_2, \hdots, \vec v_n$\}, hørende til $\lambda_1, \lambda_2, \hdots, \lambda_n$, have følgende egenskaber:
\hfill \break
\begin{enumerate}
    \item \{$\vec v_1, \vec v_2, \hdots, \vec v_n$\} former en basis for $\mathbb{R}^n$
    \item Matricen $P = (\vec v_1, \vec v_2, \hdots, \vec v_n)$ er invertibel
    \item $P^{-1}AP = \textnormal{diag}[\lambda_1, \hdots, \lambda_n]$
\end{enumerate}

\end{mytheo}

\begin{koro}{}{}
Under hypoteserne givet i \ref{th:Diagegvd}, vil løsningen til det lineære system være givet ved funktion $y(t)$, som defineret ved (brødtekst ovenfor)
\end{koro}

\section{Operatorer}

\begin{definition}[Konvergens af lineære operatorer]
En følge af lineære operatorer $T_k \in L(\mathbb{R}^n)$ konvergerer mod en anden lineær operator $T \in L(\mathbb{R}^n)$, når $k \to \infty$
$$\lim_{k\to\infty} T_k = T,$$
Hvis der $\forall \varepsilon > 0 \exists N \in \mathbb{N}: k \geq N \to ||T - T_k|| < \varepsilon$
\end{definition}

\begin{lemma}{}{}
Lad $S,T \in L(\mathbb{R}^n)$ og $x \in \mathbb{R}^n$, så gælder følgende:
\begin{enumerate}
    \item $|T(y)| \leq ||T|| |y|$
    \item $||TS|| \leq ||T|| ||S||$
    \item $||T^k|| \leq ||T||^k$ for $k = 0,1,2, \hdots$
\end{enumerate}
\end{lemma}
