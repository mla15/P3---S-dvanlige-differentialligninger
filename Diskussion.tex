\chapter{Diskussion}
I det følgende vil de to modeller af rovdyr-byttedyr systemet, som præsenteret i rapporten, blive diskuteret.\\
\hfill \break
Indledningsvis ville Lotka-Volterra's rovdyr-byttedyr system undersøges uden nogen modifikation til dette. Jævnfør de 6 antagelser fra indledningen, så vil dette system afbillede en interaktion mellem byttedyr og rovdyr i et isoleret system, og det er imidlertid ikke realistisk at sådan et system eksisterer i virkeligheden. Derfor inkluderes en modifikation til systemet, hvor en kapacitetsbegrænsing indføres for både byttedyr og rovdyr. Således vil byttedyrene ikke have ubegrænset føde, hvormed man er et skridt tættere på at kunne afbillede en virkelig situation. \\
\hfill \break
I dette projekt betragtes tilfældet, hvor systemet er modificeret med logistisk vækst, når både rovdyr og byttedyr har en kapacitetsbegrænsing
Afhængigt af værdierne af konstanterne i systemet viser stabilitetsanalysen, at de to populationer vil konvergere imod et ligevægtspunkt $(b_3^*,r_3^*)$ ,  eller et ligevægtspunkt $(b_4^*,r_4^*)$ i 1. kvadrant.\\
\hfill \break
Den signifikante forskel på de to førnævnte ligevægtspunkter, består i, at for ligevægtspunktet $(b_3^*, r_3*)$, vil rovdyrene uddø over tid uafhængigt af startpopulationen, såfremt punktet ligger på den positive b-akse. Dermed er det vigtigt, at konstanterne i systemet er nøje valgt med udgangspunkt i de to populationer af byttedyr og rovdyr, man ønsker at undersøge. Der forekommer altså markante ændringer i forhold til det ikke-modificerede system, når man begrænser de to populationer. Ligevægtspunktet går fra stabilt i det ikke-modificerede til asymptotisk stabilt, hvor rovdyrene vil uddø.\\
\hfill \break
Sammenlignes den ikke-modificerede og den modificerede model i henhold til ligevægtspunktet $(b_4^*,r_4^*)$ i 1. kvadrant, ses der også en klar forskel. I dette tilfælde vil den modificerede model, hvor begge populationer er begrænsede, have et asymptotisk ligevægtspunkt. Dette betyder, at begge populationer vil konvergere imod ligevægtspunktet, når $t \to \infty$. De to populationerne vil således ikke indgå i en uendelig cyklus, hvor størrelsen af de to populationer konstant varierer, men vil derimod nå et punkt, hvor de to arters populationer er konstant, og ikke længere påvirkes af hinanden. De to populationer er altså i ligevægt. \\
\hfill \break
%Betragtes situationen, hvor det udelukkende er byttedyrene, der er begrænset i det modificerede system, så vil $f=0$. Det bemærkes, at dette system er et specialtilfælde af systemet i \eqref{lovaeIVP}, hvorfor systemet indebærer denne begrænsning af rovdyrene, så der ikke udelukkende ses på et specialtilfælde.