\chapter{System af differentialligninger}

\section{Differentialligningssystemer}
For at kunne analysere Lotka-Volterra's model skal man kende til differentialligningssystemers opbygning og forskellige variationer deraf. Afsnittet er baseret på \citep[afsnit 1.1 og 2.1]{Perko}

\begin{definition}[Et førsteordens differentialligningssystem] \label{ikke-lin.diff}
Et \textbf{differentialligningssystem af første orden}, er et system, der kan skrives på formen
$$\dot{y}(t)=\vec{f}(t, \vec{y}(t)),$$
hvor $f: E \to \mathbb{R}^n$ og $E$ er et åbent delinterval af $\mathbb{R}^n$ og
$$\dot{y}(t) = \frac{d\vec{y(t)}}{dt} = 
\begin{bmatrix}
\frac{dy_1(t)}{dt} \\
\frac{dy_2(t)}{dt}\\
\vdots \\
\frac{dy_n(t)}{dt}
\end{bmatrix}
=
\begin{bmatrix}
f_1(t, y_1(t), y_2(t), \hdots, y_n(t))\\
f_2(t, y_1(t), y_2(t), \hdots, y_n(t))\\
\vdots \\
f_n(t, y_1(t), y_2(t), \hdots, y_n(t))
\end{bmatrix}$$
\end{definition}

Der findes indenfor differentialligningssystemer koblede systemer. Sådanne systemer defineres nu.

\begin{definition}[Ikke-koblet/koblet system] \label{ikke-kobletsystem}
Et differentialligningssystem der skrives på formen
$$\dot{y}(t)=\begin{bmatrix}
\frac{dy_1(t)}{dt} \\
\frac{dy_2(t)}{dt}\\
\vdots \\
\frac{dy_n(t)}{dt}
\end{bmatrix}=\begin{bmatrix}
f_1(t, y_1(t))\\
f_2(t, y_2(t))\\
\vdots \\
f_n(t, y_n(t))
\end{bmatrix},$$
kaldes for et \textbf{ikke-koblet system}. Et system, der ikke kan skrives på denne form, kaldes for et \textbf{koblet system}.
\end{definition}
Koefficientmatricen for et ikke-koblet system er en diagonal matrix. Bemærk at ikke-koblede lineære systemer kan løses ved hjælp af separation af variable, jævnfør sætning \ref{th:LSD}.

\begin{Example}
\textnormal{ \hfill \break
Betragt det ikke-koblede differentialligningssystem:}
\hfill \break
\begin{align*}
    y_1'(t) &= 2y_1(t)\\
    y_2'(t) &= y_2(t)\\
    y_3'(t) &= -y_3(t)
\end{align*}
\textnormal{Da kan vi skrive systemet som i definition \ref{LinSys}, hvor}
\hfill \break
\[ A =
\begin{bmatrix}
2 & 0 & 0\\
0 & 1 & 0\\
0 & 0 & -1\\
\end{bmatrix},
\]
\textnormal{hvormed systemet er ikke-koblet, hvorfor separation af variabler kan benyttes til at finde en løsning}
\begin{align*}
    y_1(t) &= c_1e^{2t}\\
    y_2(t) &= c_2e^t\\
    y_3(t) &= c_3e^{-t}
\end{align*}

\end{Example}

Af definition \ref{ikke-lin.diff} fremgår det at $\vec{f}$ kan afhænge eksplicit af $t$. Hvis det ikke er tilfældet, kaldes systemet autonomt:
\begin{definition}[Autonomitet]\label{autonom}
Et system på formen
$$\dot{y}(t)=\vec{f}(\vec{y}(t)),$$
hvor $$\vec{y}(t)=
\begin{bmatrix}
y_1(t) \\
y_2(t) \\
\vdots \\
y_n(t)
\end{bmatrix} \ \textnormal{og} \ \vec{f}(\vec{y}(t))=\begin{bmatrix}
f_1(\vec{y}(t)) \\
f_2(\vec{y}(t)) \\
\vdots \\
f_n(\vec{y}(t))
\end{bmatrix},$$ \\
kaldes for \textbf{autonomt}, og hvis det ikke er autonomt kaldes det for \textbf{ikke-autonomt}.
\end{definition}

Nu defineres lineære differentialligningssystemer.

\begin{definition}[Autonomt lineært førsteordens differentialligningssystem]\label{LinSys}
Et \textbf{autonomt lineært differentialligningssystem} er et system, der kan skrives på formen
\begin{equation} \label{linsys}
    \dot{y}(t) = A \vec{y}(t)
\end{equation}
hvor $\vec{y(t)} \in \mathbb{R}^n$ og $A$ er en $n\times n$ koefficientmatrix så:
$$\dot{y}(t)
=
\begin{bmatrix}
a_{11}(t)y_1(t)+ \hdots + a_{1n}(t)y_n(t)\\
a_{21}(t)y_1(t)+ \hdots + a_{2n(t)}y_n(t)\\
\vdots \\
a_{n1}(t)y_1(t)+ \hdots + a_{nn}(t)y_n(t)
\end{bmatrix}
=
\begin{bmatrix}
a_{11}(t) & \hdots & a_{1n}(t) \\
a_{21}(t) & \hdots & a_{2n}(t) \\
\vdots & \vdots & \vdots & \\
a_{n1}(t) & \hdots & a_{nn}(t) \\
\end{bmatrix}
\begin{bmatrix}
y_1(t)\\
y_2(t)\\
\vdots\\
y_n(t)
\end{bmatrix}$$
\end{definition}
\hfill \break
Givet en begyndelsesbetingelse: $y(0) = y_0$, kan vi finde en løsning til det autonome lineære ligningssystem, som vist ovenfor, ved hjælp af Laplace transformartion. 

\begin{Example}\textnormal{
\hfill \break
Betragt det autonome lineære koblede førsteordens differentialligningssystem:}
\begin{align*}
    y_1'(t) &= 2y_1(t)-y_2(t)\\
    y_2'(t) &= 4y_1(t)+5y_2(t)
\end{align*}

\textnormal{hvor $y_1(0)=1$ og $y_2(0)=0$}
\hfill \break

\textnormal{Vi begynder med at bruge Laplacetransformationen på begge ligninger:}

\begin{align*}
\mathcal{L}(y_1') &=2\mathcal{L}(y_1)+\mathcal{L}(y_2) \\
\mathcal{L}(y_2') &=4\mathcal{L}(y_1)+5\mathcal{L}(y_2)
\end{align*}

\textnormal{Da indsættes begyndelsesbetingelserne og vi kan omskrive:}

\begin{align*}
    sY_1(s)-1 &= 2Y_1(s) + Y_2(s) \\
    1 &= (s-2)Y_1(s) - Y_2(s) \\
    Y_1(s) &= \frac{1+Y_2(s)}{s-2}
\end{align*}

\begin{align*}
    sY_2(s)-0 &= 4Y_1(s) + 5Y2(s) \\
    0 &= -4Y_1(s)+(s-5)Y_2(s) \\
    Y_2(s) &= \frac{4Y_1(s)}{s-5}
\end{align*}

\textnormal{Vi har nu løsningerne:}

\begin{align*}
    1 &= (s-2)Y_1(s) - \frac{4Y_1(s)}{s-5} \\
    Y_1(s) &= \frac{s-5}{(s-6)(s-1)}
\end{align*}

\begin{align*}
    0 &= -4 \left( \frac{1+Y_2(s)}{s-2}\right) +(s-5)Y_2(s) \\
    Y_2(s) &= \frac{4}{(s-6)(s-1)}
\end{align*}

\textnormal{For at finde den inverse Laplacetransformation til dette kan man med fordel benytte stambrøks dekomposition, men dette vil ikke foretages i eksemplet.}

\end{Example}
\subsection{Løsning af systemer ved brug af egenværdier}
Dette afsnit er baseret på \citep[afsnit 5.4]{EP}
\begin{mytheo}{Egenværdi løsninger af $\dot{y}(t) = A \vec{y}(t)$ }{}
Lad $\lambda$ være en egenværdi for $A$-matricen i det førsteordens lineære system:
$$\dot{y}(t) = A \vec{y}(t).$$
Hvis $\vec{v}$ er en egenvektor tilhørende $\lambda$, så er 
$$\vec{y}(t)=\vec{v}e^{\lambda t}$$
en ikke triviel løsning til systemet
\end{mytheo}
\begin{proof}\\
Lad 
$$\dot{y}(t) = A \vec{y}(t)$$
være et førsteordens lineært differentialligningssystem. Sættes $\vec{y}(t)=\vec{v}e^{\lambda t}$ med den afledede $\dot{y}(t)=\lambda \vec{v}e^{\lambda t}$ ind i overstående ligning, får vi:
\begin{equation*}
\begin{split}
    \lambda \vec{v}e^{\lambda t}&=A\vec{v}e^{\lambda t} \Leftrightarrow \\
    A\vec{v}&=\lambda \vec{v}
   \end{split}
\end{equation*}
Altså vil $\vec{y}(t)=\vec{v}e^{\lambda t}$ være en ikke triviel løsning til systemet, hvis $\lambda$ er en egenværdi for $A$ med tilhørende egenvektor $\vec{v}$.
\end{proof} \\

\begin{lemma}{Lineært uafhængige løsninger}{}
Lad $\vec{y_1}, \hdots ,\vec{y_k}$ være løsninger til ligning  \eqref{linsys} i et interval $I$ og betegn løsningsrummet med $V$. Da er følgende udsagn ækvivalente:
\begin{enumerate}
    \item $\vec{y_1}, \hdots ,\vec{y_k}$ er lineært uafhængige i $V$
    \item $\vec{y_1}(t), \hdots ,\vec{y_k}(t)$ er lineært uafhængige i $\mathbb{R}^n$ for alle $t \in I$
    \item Der eksisterer et $t_0$, så $\vec{y_1}(t_0), \hdots ,\vec{y_k}(t_0)$ er lineært uafhængige i $\mathbb{R}^n$ for $t_0 \in I$
\end{enumerate}
\end{lemma}
\begin{proof}
Vi vil vise at $1. \Rightarrow 2. \Rightarrow 3. \Rightarrow 1. $ \\ \hfill \break
$1. \Rightarrow 2.$: \\ Vi vil bevise dette ved kontraposition, altså vil vi vise, at hvis $\vec{y_1}(t), \hdots ,\vec{y_k}(t)$ er lineært afhængige i $\mathbb{R}^n$, så er $\vec{y_1}, \hdots ,\vec{y_k}$ lineært afhængige i $V$. \\
\hfill \break
Antag at der eksisterer et $t_0$ således, at $\vec{y_1}(t_0), \hdots ,\vec{y_k}(t_0)$ er lineært afhængige i $\mathbb{R}^n$. Så findes der koefficienter $\lambda_i \in \mathbb{R}$, som ikke alle er nul således, at $\sum_{i=1}^k \lambda_i\vec{y_i}(t_0)=\vec{0}$. Betragtes de to funktioner:
\begin{equation*}
    \vec{y}= \sum_{i=1}^k \lambda_i\vec{y_i} \ \ \ \textnormal{og} \ \ \ \vec{y}=\vec{0}
\end{equation*}
ses det, at de begge er løsninger til $IVP$'et $\dot{y}=A\vec{y}(t), \ y(t_0)=\vec{0}$, og de er dermed lig hinanden grundet $EESFS$ (se næste side). Dette vil altså sige, at $\sum_{i=1}^k \lambda_i\vec{y_i}= \vec{0}$, og funktionerne $\vec{y_1}, \hdots ,\vec{y_k}$ er dermed lineært afhængige i $V$. \\ 
\hfill \break
$2. \Rightarrow 3.$: \\ Trivielt \\ 
\hfill \break
$3. \Rightarrow 1.$: \\
Antag at $\sum_{i=1}^k \lambda_i\vec{y_i}=\vec{0}$, det vil sige, at $\sum_{i=1}^k \lambda_i\vec{y_i}(t)=\vec{0}$ for alle $t \in I$. Sæt $t=t_0$. Så får vi, at $\sum_{i=1}^k \lambda_i\vec{y_i}(t_0)=\vec{0}$, og da $\vec{y_1}(t_0), \hdots ,\vec{y_k}(t_0)$ er lineært uafhængige, følger det, at $ \lambda_1 = \hdots = \lambda_k = 0$.
\end{proof}

Som afsluttende bemærkning på teorien om differentialligningssystemer bemærkes det, at EES kan formuleres for systemer, jævnfør \citep[s. 144]{Hirsch}.\\

\begin{mytheo}{Eksistens- og entydighedssætningen for systemer (EESFS)}{EESFS}
Lad 
\begin{equation*}
    \begin{cases}
      \dot{y}(t)&=\vec{f}(t,\vec{y}(t))\\
      \vec{y}(t_0)&=\vec{y_0}
    \end{cases}
\end{equation*}
være et førsteordens IVP, hvor: 

\begin{enumerate}
    \item $\vec{f}\colon A \to   \mathbb{R}^n$, hvor $A\subseteq\mathbb{R}^{n+1}$ er en åben delmængde af $\mathbb{R}^{n+1}$.
    \item $\frac{\partial\vec{f}}{\partial \vec{y}}$ eksisterer og er kontinuert på A.
    \item $(t_0,\vec{y_0}) \in A$.
\end{enumerate}

Da findes til hver omegn $]t_0-\delta,t_0+\delta[ \ \times B_r(y_0)$ et interval $]\alpha,\beta[ \ \subseteq \ ]t_0-\delta,t_0+\delta[$, hvorpå en entydigt bestemt løsning $\phi\colon \ ]\alpha,\beta[ \ \to \mathbb{R}^n$ er defineret.
\end{mytheo}
Det bemærkes, at ovenstående sætning betyder, at løsningsmængden\\
$M=\{\phi| \ \phi \text{ er en løsning til det givne IVP}\}$ ikke er tom.
\begin{definition}[Maksimalt definitionsinterval]
Lad et IVP som i sætningen ovenfor være givet, hvortil der findes en entydigt bestemt løsning, $\phi(t)$, på intervallet $\ ]\alpha,\beta [\ \subseteq \ ]t_0-\delta,t_0+\delta[\ $, hvor $t_0\in \ ]\alpha,\beta[\ $.\\
Sæt nu $$a_1=\inf\{\tau|\text{der eksisterer en løsning}, y(t)\in M, \text{på} \ ]\tau , t_0] \}$$
$$a_2=\sup\{\tau|\text{der eksisterer en løsning}, y(t)\in M, \text{på } [t_0 , \tau[ \}$$
Da kaldes $]a_1,a_2[$ det maksimale definitionsinterval.
\end{definition}
I ovenstående definition er $y(t)$ nødvendigvis defineret på et større interval, $I\supseteq\ ]\tau , t_0]$ for $a_1$ og $J\supseteq [t_0 , \tau[$ for $a_2$, men resten af definitionsintervallet er uden betydning for definitionen.
\begin{lemma}{Maksimal løsning}{LMD}
Lad $]a_1,a_2[$ være det maksimale definitionsinterval for et givet IVP. Der findes en entydigt bestemt løsning, $\phi(t)$, defineret på intervallet $]a_1,a_2[$.
\end{lemma}

\begin{proof}\\
Beviset føres som et konstruktivt bevis, hvor vi konstruerer $\phi(t)$ og viser, at denne løsning er entydig.\\ \hfill \break
   Lad $\tau_0\in \ ]a_1,a_2[$ være givet. Antag, at $a_1<\tau_0\leq t_0$. Fordi $a_1<\tau_0$, så findes en løsning $\phi_0(t)$ defineret på $\left]\tau_1,t_0\right]$ for et passende $\tau_1\in \ \left]a_1,\tau_0\right[$. Sæt nu $\phi(\tau_0) = \phi_0(\tau_0)$. Dette er tilladeligt, da man med en tilsvarende anden løsning $\tilde{\phi}_0$ defineret på $ \left]\tilde{\tau}_1, t_0\right]$, hvor $\tilde{\tau}_1\in \left]a_1,\tau_0\right[$,  har, at $\tilde{\phi}_0(\tau_0)=\phi_0(\tau_0)$. Pér sætning \ref{th:EES} ved vi nemlig, at der eksisterer en omegn af $t_0$, hvor $\phi_0(t) = \tilde{\phi}_0(t)$. Derfor er $T = \{\tau | \phi_0(t) = \tilde{\phi}_0(t) \ \forall t \in \ ]\tau, t_0] \}$ ikke tom. Pér konstruktion er $T$ nedadtil begrænset af $\max\{\tau_1,\tilde{\tau}_1\}$, så vi kan sætte $\tau_2 = \inf T$. Da $\phi_0$ og $ \tilde{\phi}_0$ begge er kontinuerte, så er $\tau_2 \in T$. Det er klart, at $\tau_2 \geq \max\{\tau_1,\tilde{\tau}_1\}$. Hvis $\tau_2 = \max\{\tau_1,\tilde{\tau}_1\}$, så er $\phi_0(\tau_0) = \tilde{\phi}_0(\tau_0)$ som ønsket. \\ \hfill \break
   Antag derfor, at $\tau_2>\max\{\tau_1,\tilde{\tau}_1\}$. Nu anvendes EES på følgende IVP: $y' = f(t,y)$, $y(\tau_2) = \phi(\tau_2)$. Pér EES findes der et interval $]\alpha, \beta[\ni\tau_2$, hvorpå der eksisterer en entydigt bestemt løsning. Specielt findes der et interval $]\tilde{\alpha},\tau_2]$ med $\tilde{\alpha}=\max\{\tau_1,\tilde{\tau_1}\}$, hvor $\phi_0(t) = \tilde{\phi}_0(t)$, i modstrid med at $\tau_2=\inf T$. Dermed findes der ingen værdier af $t\in \ ]\max\{\tau_1,\tilde{\tau}_1\},t_0]$ sådan at $\phi_0(t) \neq \tilde{\phi}_0(t)$.\\ \hfill \break Således er $\phi(\tau_0)$ entydigt bestemt, og da $\tau_0\in \ ]a_1,t_0]$ var arbitrær, er $\phi(t)$ veldefineret for alle $t\in  \ ]a_1,t_0]$. Vi kan konstruere $\phi(t)$ på intervallet $[t_0,a_2[$ helt analogt. Nu mangler vi blot at vise, at den således konstruerede funktion, $\phi(t)$, er en løsning til det givne IVP. Lad som før $\tau_0\in  \ ]a_1,a_2[$ være givet. Som vist tidligere i beviset findes en løsning $\tilde{\phi}(t)$ defineret på en omegn, $]\tau_0-\delta,\tau_0+\delta[$, af $\tau_0$. I denne omegn stemmer $\phi(t)$ overens med $\tilde{\phi}(t)$. Da $\tilde{\phi}(t)$ er en løsning, er den differentiabel på $]\tau_0-\delta,\tau_0+\delta[$, og dermed er $\phi(t)$ differentiabel på samme interval. Da $\tau_0$ var arbitrær, er $\phi(t)$ differentiabel på hele $]a_1,a_2[$. Det er klart, at $\phi(t_0)=\vec{y}_0$, fordi $\phi(t)$ er konstrueret, så den stemmer overens med løsninger i en omegn af $t_0$. Dermed er $\phi(t)$ en løsning. 
\end{proof}\\ \hfill \break
Løsningen $\phi(t)$ kaldes den maksimale løsning. Det bemærkes, at Pér konstruktion af $\phi(t)$ er alle andre løsninger restriktioner af $\phi(t)$: Hvis $\tilde{\phi}\colon\ ]\alpha,\beta[ \ \to \mathbb{R}^n$ er en løsning, så er $]\alpha,\beta[ \ \subseteq \ ]a_1,a_2[$ og $\tilde{\phi}(t)=\phi(t) \ \forall t\in \ ]\alpha,\beta[$, jævnfør beviset.
\begin{Example}
\textnormal{\hfill \break Betragt følgende IVP:}
$$\begin{cases}
      y'(t)&=1+y(t)^2\\
      y(0)&=0
    \end{cases}$$
\textnormal{Dette IVP har maksimal løsning $y(t)=\tan(t)$ defineret på $]-\frac{\pi}{2},\frac{\pi}{2}[$.}
\end{Example}
%Givet et IVP: $y(0) = y_0$, kan vi finde en løsning til det lineære ligningssystem, som vist ovenfor, givet ved:
%$$y(t) = e^{At}y_0$$
%Ved at se på definitionen for lineære differentialligningssystemer, kan det udledes, at $e^{AT}$ er en $n\times n$ matrix, som kan findes ved hjælp af taylorpolynomier.%(forklaring? xD Hvordan kommer vi egentlig frem til $e^{AT}$?) Smid en reference til prop: 2.4.4.
\begin{mytheo}{Løsningskurverne forlader enhver kompakt delmængde. \citep[Sætning 3 s. 42]{Svensk}}{}\label{forladkompakt}
Lad et IVP være givet som i EESFS. Lad $]a_1,a_2[$ være det maksimale definitionsinterval, og lad $K\subset A$ være en kompakt delmængde af definitionsmængden for $\vec{f}$. For den maksimale løsning, $\phi(t)$, til det givne IVP, vil punktet $(t,\phi(t))$ ligge udenfor $K$, hvis $t$ er tilstrækkelig tæt på $a_1$ eller $a_2$.
\end{mytheo}
Sætningen bevises ikke her i projektet, men et bevis findes i \citep[s. 43-44]{Svensk}. \\ \hfill \break
