\chapter{Differentialligninger}
I det følgende kapitel vil teorien for differentialligninger og løsninger dertil blive gennemgået, da en forståelse af denne er nødvendig for at kunne udvide Lotka-Volterra's model og vurdere denne. 

\section{Sædvanlige differentialligninger}
Først defineres en sædvanlig differentialligning, hvori $I$ er et interval i $\mathbb{R}$:
\begin{definition}[Sædvanlig differentialligning (ODE)]\label{LoesningODE}
En \textbf{sædvanlig differentialligning} er en ligning på formen: \hfill \break
$$G(t,y(t),y'(t), \hdots , y^{(n)}(t))=0,$$
hvor:
\begin{itemize}
    \item $G:A \to \mathbb{R}$ er en funktion, hvor $A \subseteq \mathbb{R}^{n+2}$ er en åben delmængde af $\mathbb{R}^{n+2}$.
    \item $y(t)$ er en funktion $y\colon I\to \mathbb{R}$, hvor $I\subseteq \mathbb{R}$ er et interval.
\end{itemize}.
\end{definition}

Ordenen, $n$, af en ODE angiver, at den $n$'te afledte af funktionen $y(t)$ er den højest afledte. 
En ODE kan løses ved at bestemme en funktion, der opfylder følgende:

\begin{definition} [Løsning til ODE]
En \textbf{løsning til en ODE} af $n$'te orden, er en funktion, $\phi: I\to \mathbb{R}$, hvor $I\subseteq \mathbb{R}$ er et interval, som opfylder
\begin{itemize}
    \item $\phi(t)$ er $n$ gange differentiabel
    \item $\forall t\in I: (t,\phi(t),\phi'(t),\hdots,\phi^n(t))\in A$
    \item $\forall t\in I: G(t, \phi(t),\phi'(t),\hdots,\phi^n(t))=0$
\end{itemize}

En \textbf{partikulær løsning} er en bestemt løsning, mens den \textbf{fuldstændige løsning} er mængden af alle løsninger til ODE.
\end{definition}

\begin{definition}[Homogenitet]
En ODE kaldes \textbf{homogen}, hvis den kan skrives på formen: 
$$G(y(t),y'(t), \hdots, y^{(n-1)}(t),y^{(n)}(t))=0$$ 
Altså hvis $G$ ikke afhænger eksplicit af den uafhængige variabel, $t$. En \textbf{inhomogen} ODE er på formen: 
$$G(y(t),y'(t),\hdots, y^{(n-1)}(t),y^{(n)}(t))+f(t)=0$$
I dette tilfælde kaldes $f(t)$ en \textbf{inhomogenitet}. En ODE der indeholder en inhomogenitet kaldes \textbf{inhomogen}, mens en ODE, der ikke indeholder en inhomogenitet, kaldes \textbf{homogen}.
\end{definition}
Til den inhomogene ODE, $G(y(t),y'(t),...,y^{(n)}(t))=f(t)$, kalder vi ligningen, $G(y(t),y'(t),...,y^{(n)}(t))=0$, den associerede homogene ODE. 

Ved en ODE er det væsentligt at skelne mellem lineære og ikke-lineære ODEs, da disse har forskellige løsningsmetoder og beskriver forskellige situationer.

\begin{definition}[Sædvanlig lineær differentialligning (OLDE)]\label{OLDE}En ODE af $n$'te orden kaldes lineær, hvis de afledte op til orden $n$ indgår i form af en linearkombination. Altså: \\ 
$$a_{n}(t)y^{(n)}(t)  + a_{n-1}(t)y^{(n-1)}(t)+ \hdots + a_{1}(t)y'(t) + a_{0}(t)y(t) = f(t)$$

hvis koefficienten $a_n(t)\neq 0 \ \forall t$, kan ligningen normeres til:

$$y^{(n)}(t)+P_{n-1}(t)y^{(n-1)}(t)+\hdots +P_0(t)y(t)=Q(t)$$ 

Her er: $P_0(t)=\frac{a_0(t)}{a_n(t)} , P_1(t)=\frac{a_1(t)}{a_n(t)}, \hdots, P_{n-1}(t)=\frac{a_{n-1}(t)}{a_n(t)}$ og $Q(t)=\frac{f(t)}{a_n(t)}$
\end{definition}

%\begin{mytheo}{Generel løsning for homogene OLDE}{}
%    Lad $y_1,y_2,\hdots,y_n$ være $n$ lineære uafhængige løsninger til den homogene ligning:
%    \begin{equation}\label{glhomo}
%        a_n(t)y^{(n)}+\hdots +a_1(t)y'+a_0(t)y=0
%    \end{equation}
%    på et åbent interval $I$, hvor $p_i$ er sammenhængende.
%    Hvis $Y$ er en hvilken som helst løsning til \eqref{glhomo}, da vil der eksistere $c_1,c_2,\hdots %c_n$ således at:
%    $$Y(t)=c_1y_1(t)+c_2y_2(t)+\hdots+c_ny_n(t)$$
%    for alle $x\in I$
%\end{mytheo}
%Dermed er alle løsninger til en homogen $n$'te orden OLDE en lineær kombination, 
%$$y=c_1y_1+c_2y_2+\hdots+c_ny_n$$,
%af enhver $n$ given lineær uafhængig løsning. Baseret på dette kalder vi en sådan lineær kombination %en generel løsning for en homogen OLDE.

\begin{mytheo}{Den fuldstændige løsning til en inhomogen OLDE\\ i \citep[Theorem 5, s. 122]{EP}}{}
En inhomogen OLDE på formen: 
\begin{equation}\label{linhomo}
a_n(t)y^{(n)}(t)+\hdots +a_1(t)y'(t)+a_0(t)y(t)=f(t), \enspace t\in I
\end{equation}
med den associerede homogene OLDE:
\begin{equation}\label{lhomo}
a_n(t)y^{(n)}(t)+\hdots +a_1(t)y'(t)+a_0(t)y(t)=0, \enspace t\in I
\end{equation}
har den fuldstændige løsning:
\begin{equation}
    L_{inhomo}= \{ y(t) | \exists y_f\colon y(t)=y_f(t)+y_p(t) \forall t \in I \}
\end{equation}
hvor $y_p$ er en partikulær løsning til ligningen, og $y_f$ er en vilkårlig løsning til den associerede homogene OLDE. 
\end{mytheo}

\begin{proof}\\
Antag at en partikulær løsning $y_p$ til ligning $\eqref{linhomo}$ er kendt og, at $Y$ er en vilkårlig anden løsning til ligning $\eqref{linhomo}$. Hvis $y_f=Y-y_p$, så får vi ved at indsætte $y_f$ i differentialligningen:
\begin{align*}
    a_n(t)y_f^{(n)}+\hdots +a_1(t)y_f'+a_0(t)y_f&=
    (a_n(t)Y^{(n)}+\hdots +a_1(t)Y'+a_0(t)Y)\\
    &-(a_n(t)y_p^{(n)}+\hdots +a_1(t)y_p'+a_0(t)y_p)\\
    &=F(t)-F(t)=0
\end{align*}
Dermed er $y_f=Y-y_p$ en løsning til den associerede homogene ligning \eqref{lhomo}. Løsningen kan dertil  omskrives til
$$Y=y_f+y_p$$

\hfill \break
Antag at $a_n(t) \neq 0, \forall t \in I$, da kan ligning \eqref{lhomo} normeres til 
$$y^{(n)} + P_{n-1}(t)y^{(n-1)} + \hdots + P_0(t)y = 0.$$
Det ses nu, at $y_p+y_f$ altid er en løsning til ligning \eqref{linhomo}, da det følger af \citep[Theorem 4, s. 120]{EP}, at der vil eksiterere tal $c_1, c_2, \hdots, c_n$ sådan, at
$$y_f=c_1y_1+c_2y_2+ \hdots +c_ny_n,$$
hvor $y_1, y_2, \hdots, y_n$ er lineært uafhængige løsninger til ligning \eqref{lhomo}. Pér \citep[Theorem 2, s. 114]{EP} vides det, at kontinuerte funktioner på et åbent interval $I$, hvorpå $a$ er en værdi. Givet $n$ tal $b_0, b_1, \hdots, b_{n-1}$, så har ligning \eqref{lhomo} en entydig løsning på intervallet $I$, der opfylder de $n$ mulige begyndelsesværdier:
$$y(a)=b_0, \ y'(a)=b_1, \ y^{(n-1)}(a)=b_{(n-1)}.$$


Dermed er det bevist, at en generel løsning til ligning \eqref{linhomo} er summen af en vilkårlig løsning $y_f$ og den partikulære løsning $y_p$ til ligning \eqref{lhomo}.
\end{proof}

\section{Begyndelsesværdiproblemer}
Begyndelsesværdiproblemer omfatter både enkeltstående differentialligninger og systemer af differentialligninger, hvorfor det vil blive beskrevet i det følgende.
\begin{definition}[Begyndelsesværdiproblem (IVP)]
Et begyndelsesværdiproblem (IVP) er en ODE med en betingelse, som $y, y',\hdots, y^{(n)}$ skal overholde for en specifik værdi af den uafhængige variabel, altså:
\begin{align*}
    y(t_0) &= y_0 \\
    y'(t_0) &= y_1 \\
    &\vdots \\
    y^{(n)}(t_0) &= y_n
\end{align*}
\end{definition}

\begin{Example}\textbf{IVP}\hfill \break
\textnormal{Betragt følgende IVP:}
\hfill \break
\begin{equation}\label{IVPODE}
    y(t)+y''(t)=0, t \in \mathbb{R}
\end{equation}
\textnormal{med begyndeslesværdibetingelserne}
\begin{equation*}
    \begin{cases}
    y(0)&=1\\
    y'(0)&=0
    \end{cases}
\end{equation*}
\hfill \break
\textnormal{Det ses hurtigt, at $y(t)=\cos(t)$ er en løsning til dette IVP og dette kan efterprøves. Først på ligning \eqref{IVPODE}:}
\begin{align*}
    y(t)+y''(t)&=0 \\
    \cos(t)+(-\cos(t))&=0 \\ 
\end{align*}
\textnormal{og derefter på begyndelsesværdibetingelserne:}
\begin{align*}
      y(0)=\cos(0)&=1\\
    y'(0)=-\sin(0)&=0\\
\end{align*}
\textnormal{Dermed er dette IVP løst.}
\end{Example}

\begin{mytheo}{Eksistens- og entydighedssætningen (EES)}{EES}
Lad 
\begin{equation*}
    \begin{cases}
      y'(t)&=f(t,y(t))\\
      y(t_0)&=y_0
    \end{cases}
\end{equation*}
være et førsteordens IVP, hvor: 

\begin{enumerate}
    \item $f: A \rightarrow \mathbb{R}$, hvor $A\subseteq \mathbb{R}^2$ er en åben delmængde af $\mathbb{R}^2$.
    \item $\frac{\partial f}{\partial y}$ eksisterer og er kontinuert.
    \item $(t_0,y_0) \in A$.
\end{enumerate}
Da findes til hver omegn $]t_0-\delta,t_0+\delta[ \ \times \ ]y_0-\tilde{\delta},y_0+\tilde{\delta}[$ et interval $]\alpha,\beta[ \ \subseteq \ ]t_0-\delta,t_0+\delta[$, hvorpå en entydigt bestemt løsning $\phi\colon \ ]\alpha,\beta[ \ \to \mathbb{R}^n$ er defineret.
\end{mytheo}

Til at bevise denne sætning kan man benytte Lipschitz betingelser samt Banachs fix-punkt sætning, men disse vil ikke blive beskrevet yderligere i denne rapport, da fokus, for projektet, ligger på at undersøge en udvidelse af Lotka-Volterra's model. Samtidig er beviset og forudsætninger dertil omfattende.

\section{Førsteordens differentialligninger}
I det følgende vil vi se på førsteordens differentialligninger samt løsningsmetoder dertil, da disse er byggesten til Lotka-Volterra's model.
Følgende afsnit er baseret på \citep[kapitel 2.2]{JAB}, hvis andet ikke er noteret.
\subsection{Separable differentialligninger}
Følgende beskriver en central metode til omskrivning af differentialligninger, som er anvendelig, når en løsning søges.
\begin{definition}[Separabel differentialligning]
En førsteordens ODE kaldes \textbf{separabel}, hvis den kan skrives på formen, $$y'(t)=g(t)p(y(t))$$ hvor $g(t)$ kun afhænger af $t$, og $p(y(t))$ kun afhænger af $y(t)$.
\end{definition}

\begin{mytheo}{Løsning til separable differentialligninger}{LSD}
Lad $p$ være en kontinuert funktion og $p(y_0) \neq 0$. Lad følgende være givet:
\begin{equation}\label{LSD}
    \begin{aligned}
    &y'(t)=g(t)p(y(t))\\ 
    &y(t_0)=y_0,
    \end{aligned}
\end{equation}
så eksisterer der en løsning, $y(t)$:
$$y(t)=H^{-1} \left ( H(y_0)+\int_{t_0}^tg(s)ds \right ),$$
hvor $H$ er en stamfunktion til $\frac{1}{p(y(t))}$.
\end{mytheo}

\begin{proof}\\
Antag at $y(t)$ er en løsning til ligning \eqref{LSD}.
\hfill \break
Da $p(y_0) \neq 0$, så er $p(y(t)) \neq 0$ for $t$ i en omegn af $t_0$, da $p$ og $y(t)$ er kontinuert. Dvs. at $\frac{1}{p(y(t))}$ er defineret i denne omegn af $t_0$.
\hfill \break
Vi starter med at skrive ligning i \eqref{LSD} på formen:
\begin{equation*}
    \frac{1}{p(y(t))}y'(t)= g(t)
\end{equation*}
Da $\frac{1}{p(y(t))}$ er en kontinuert funktion, så er der en stamfunktion $H(y(t))$ således at $H'(y(t))=\frac{1}{p(y(t))}$, så har vi: 
$$H'(y(t)) y'(t)=g(t)$$
Ved at anvende kædereglen, som siger $f'(g(x))g'(x)=(f(g(x)))'$, får vi:
$$(H(y(t)))'=g(t)$$
og ved integration får vi:
\begin{align*}
    \int_{t_0}^t(H(y(s)))'ds&=\int_{t_0}^tg(s)ds\\
    H(y(t))-H(y(t_0))&=\int_{t_0}^tg(s)ds
\end{align*}
Da vi ved fra ligning \eqref{LSD} at $y(t_0)=y_0$, får vi:
$$H(y(t))=H(y_0)+\int_{t_0}^tg(s)ds$$
Nu er det opgaven at isolere $y(t)$ for at finde vores løsning til ligning \eqref{LSD}.
Hvis den inverse funktion til $H$ eksisterer, så:
$$y(t)=H^{-1} \left ( H(y_0)+\int_{t_0}^tg(s)ds \right )$$

Omvendt skal vi også vise at
$$t,t_0 \mapsto H^{-1} \left ( H(y_0)+\int_{t_0}^tg(s)ds \right ) $$ er en løsning til ligning \eqref{LSD}.
Vi starter med at indsætte $t_0$ for at se, om den stemmer overens: 
\begin{align*}
    y(t_0)&=H^{-1} \left ( H(y_0)+\int_{t_0}^{t_0}g(s)ds \right )\\
    &=H^{-1} (H(y_0))\\
    &=y_0
\end{align*}
Det passer med ligning \eqref{LSD}. Nu kigger vi så på $t$, hvor vi starter med at tage $H$ af begge sider:
$$H(y(t))=H(y_0)+\int_{t_0}^tg(s)ds$$
Nu får vi ved at differentiere leddene:
$$(H(y(t)))'=g(t)$$
Her anvender vi igen kædereglen, hvor vi får:
$$H'(y(t)) y'(t)=g(t)$$
Vi ved at $H(y(t))$ er en stamfunktion, således at $H'(y(t))=\frac{1}{p(y(t))}$, som vi indsætter:
$$\frac{1}{p(y(t))}y'(t)=g(t) \ \Rightarrow \ y'(t)=g(t)p(y(t))$$
Det passer med ligning \eqref{LSD}.
\end{proof}\\

Et eksempel på en løsning til separable differentialligninger ses i afsnit \ref{lovaeg}.  


%%\begin{Example}
%%Antag at vi har en førsteordens OLDE:\\
%%\hfill \break
%%\centerline{$ty'(t) + 2y(t) = t^2 - t + 1$}
%%\hfill \break
%%Hvis vi dividerer igennem med 2t fås:\\
%%\hfill \break
%%\centerline{$y'(t) + \frac{2}{t}y(t) = t + -1 \frac{1}{t}$}
%%\hfill \break
%%\centerline{$\int P(t)dt = \int \frac{2}{t}dt = e^2ln(t)}
%%\hfill \break
%%Da har vi integrationsfaktoren, $\mu = e^{2ln|t|} = e^{ln(t^2)} = t^2$\\
%%\hfill \break
%%\centerline{$t^2 \frac{dy}{dt} + 2ty = \frac{3t}{2}$}
%%\hfill \break
%%\centerline{$\frac{d}{dt}(t^2y) = \frac{3t}{2}$ $\rightarrow$ $t^2y = \int \frac{3t}{2}dt = \frac{3}{4}t^2 + K$}
%%\hfill \break
%%Da får vi: \\
%%\hfill \break
%%\centerline{$y = \frac{3}{4}t^2t^{-2} = \frac{3}{4} + Kt^{-2}$}
%%\end{Example}
%% kilde: http://tutorial.math.lamar.edu/Classes/DE/Linear.aspx 

\section{Andenordens differentialligninger}\label{andendiff}
Følgende afsnit inkluderes som følge af studieordningen, som sætter projektets rammer, men ligeledes fordi, man kan omskrive andenordens ODEs til koblede systemer af differentialligninger. Afsnittet er baseret på \citep[afsnit 4.2]{JAB}.
\subsection{Homogene lineære andenordens ODE med konstante koefficienter}
En homogen lineær andenordens ODE med konstante koefficienter er en ODE på formen: \hfill \break
\begin{equation}
\label{homlinandord}
    a_2y''(t)+a_1y'(t)+a_0y(t)=0
\end{equation} \hfill \break
Hvor $a_2,a_1,a_0\in \mathbb{R}$ er de konstante koefficienter, og $a_2\neq 0$. \hfill \break

\begin{definition}[Den karakteristiske ligning]
Ligningen 
$$a_2r^2+a_1r+a_0=0$$
hvor $a_2, a_1$ og $a_0$ er koefficienterne fra ligning \eqref{homlinandord}, kaldes \textbf{den karakteristiske ligning} for ligning \eqref{homlinandord}.
\end{definition} 
\hfill \break
Det bemærkes at for en inhomogen andenordens ODE på formen
$$a_2r^2+a_1r+a_0=f(t)$$
kaldes ligningen i ovenstående definition også for den tilhørende karakteristiske ligning.
Den  karakteristiske ligning er et andengradspolynomium, og de værdier af $r$, som opfylder denne ligning, er vigtige, som det fremgår af følgende proposition: \hfill \break
\begin{prop}{Løsning af homogen andenordens (ODE)}{losprop2ordhom}\label{losprop2ordhom}
Funktionen $y(t)=e^{rt}$ er en løsning til ligning \eqref{homlinandord}, hvis og kun hvis $r$ er en løsning til den karakteristiske ligning for ligning \eqref{homlinandord}.
\end{prop}
\hfill \break
\begin{proof}\\
Vi sætter $y=e^{rt}$ og bemærker, at $y'=re^{rt}$ samt $y''=r^2e^{rt}$. Nu skrives (\ref{homlinandord}): \hfill \break
 $$a_2r^2e^{rt}+a_1re^{rt}+a_0e^{rt}=0 \Leftrightarrow$$ 
 $$e^{rt}(a_2r^2+a_1r+a_0)=0$$
 Da $e^{rt}>0$ må det ifølge nulreglen gælde at: \hfill \break
 $$a_2r^2+a_1r+a_0=0$$
\end{proof} \\ 
\hfill \break

Vi har altså en simpel metode til at finde en partikulær løsning til ligning \eqref{homlinandord}:\hfill \break

\begin{Example}\label{ekspartlos} \textbf{Partikulær løsning}\hfill \break
\textnormal{Betragt følgende ODE:}\\ \hfill \break
\centerline{$-y''(t)+2y'(t)+3y(t)=0$}\\ \hfill \break
\textnormal{Nu kan vi finde en partikulær løsning ved først at finde rødderne til den karakteristiske ligning:}\\ \hfill \break
\centerline{$-r^2+2r+3=0$}\\ \hfill \break
\textnormal{Rødderne er i dette tilfælde:}\\ \hfill \break
\centerline{$r_1=\frac{-2+\sqrt{2^2-4\cdot (-1)\cdot 3}}{2\cdot (-1)}=-1$ \textnormal{og} $r_2=\frac{-2-\sqrt{2^2-4\cdot (-1)\cdot 3}}{2\cdot (-1)}=3$} \\ \hfill \break
\textnormal{Vælger vi for eksempel $r_2$ og sætter $y(t)=e^{3t}$, kan vi tjekke, at det er en partikulær løsning til ovenstående ODE:}\\ \hfill \break
\centerline{$(-1)\cdot3^2e^{3t}+2\cdot 3e^{3t}+3e^{3t}=0e^{3t}=0$}\\ \hfill \break
\textnormal{På samme måde kan det vises at $e^{(-1)t}$ også er en partikulær løsning.}
\end{Example}\hfill \break

I det følgende, som er baseret på \citep{2ordhom}, ønsker vi en metode til at finde den fuldstændige løsning til ligning \eqref{homlinandord}. Først skal vi vise to propositioner:\hfill \break
\begin{prop}{Linearkombination af løsninger}
HHvis $y_1(t)$ og $y_2(t)$ begge er løsninger til ligning \eqref{homlinandord}, og $k_1,k_2\in \mathbb{R}$ er arbitrære konstanter, så er \hfill \break
$$y(t)=k_1y_1(t)+k_2y_2(t),$$ \hfill \break
også en løsning til ligning \eqref{homlinandord}.
\end{prop}
\hfill \break
\begin{proof}\\
Lad $y_1$ og $y_2$ være løsninger til ligning \eqref{homlinandord}, og lad $k_1,k_2\in \mathbb{R}$ være givet. Nu tjekker vi om $k_1y_1(t)+k_2y_2(t)$ er en løsning til ligning \eqref{homlinandord}:
\hfill \break
\begin{align*}
a_2y''(t)+a_1y'(t)+a_0y(t)&=a_2(k_1y_1(t)+k_2y_2(t))''+a_1(k_1y_1(t)+k_2y_2(t))'+a_0(k_1y_1(t)+k_2y_2(t)) \\
&=a_2(k_1y_1''(t)+k_2y_2''(t))+a_1(k_1y_1'(t)+k_2y_2'(t))+a_0(k_1y_1(t)+k_2y_2(t)) \\
&=a_2k_1y_1''(t)+a_2k_2y_2''(t)+a_1k_1y_1'(t)+a_1k_2y_2'(t)+a_0k_1y_1(t)+a_0k_2y_2(t) \\
&=k_1(a_2y_1''(t)+a_1y_1'(t)+a_0y_1(t))+k_2(a_2y_2''(t)+a_1y_2'(t)+a_0y_2(t))
\end{align*}
\hfill \break
Da $y_1(t)$ er en løsning, ved vi at $a_2y_1''(t)+a_1y_1'(t)+a_0y_1(t)=0$, og da $y_2(t)$ er en løsning, ved vi at $a_2y_2''(t)+a_1y_2'(t)+a_0y_2(t)=0$. Altså har vi:\hfill \break
$$k_1(a_2y_1''(t)+a_1y_1'(t)+a_0y_1(t))+k_2(a_2y_2''(t)+a_1y_2'(t)+a_0y_2(t))=k_1\cdot 0+k_2\cdot 0=0$$\hfill \break
Dermed er $k_1y_1(t)+k_2y_2(t)$ en løsning til ligning \eqref{homlinandord}.
\end{proof}\\
\hfill \break
\begin{mytheo}{Fuldstændig løsning}{}
Hvis $y_1(t)$ og $y_2(t)$ er lineært uafhængige partikulære løsninger til ligning \eqref{homlinandord}, og $k_1 \in \mathbb{R}$ og $k_2\in \mathbb{R}$ er arbitrære konstanter, så er den fuldstændige løsning til ligning \eqref{homlinandord} givet ved: \hfill \break
$$y(t)=k_1y_1(t)+k_2y_2(t)$$
\end{mytheo}
\hfill \break
Vi har vist at $y(t)$ er en løsning, men vi mangler at vise, at enhver løsning kan skrives på den form. Beviset bygger på eksistens og entydighedssætningen. Beviset føres ikke i dette projekt, men findes i \citep[s. 221-225]{JAB}.\\
\hfill \break
Vi skal altså kende to lineært uafhængige partikulære løsninger for at kunne skrive den fuldstændige løsning op. Proposition \ref{losprop2ordhom} siger, at vi kan finde partikulære løsninger ved at løse den karakteristiske ligning. Som bekendt kan et andengradspolynomium enten have to forskellige reelle rødder, én reel rod eller to forskellige komplekse rødder. Vi viser for hvert tilfælde, hvordan den fuldstændige løsning findes. \\ \hfill \break
\textbf{Tilfælde 1}\hfill \break
Hvis den karakteristiske ligning har to forskellige reelle rødder, $r_1$ og $r_2$, så er løsningerne $e^{r_1t}$ og $e^{r_2t}$ lineært uafhængige. Det ses let, da der ikke findes nogen konstant, $k\in \mathbb{R}$, så $e^{r_1t}=ke^{r_2t}$ for alle $t$. Den fuldstændige løsning kan derfor skrives: \hfill \break
$$y(t)=k_1e^{r_1t}+k_2e^{r_2t}$$\hfill \break
Hvor $k_1,k_2\in \mathbb{R}$ er arbitrære konstanter.\\ \hfill \break
\begin{Example}\textbf{To reelle rødder}\hfill \break
\textnormal{Betragt ligningen:} \hfill \break
\centerline{$-y''(t)+2y'(t)+3y(t)=0$}\\ \hfill \break
\textnormal{fra eksempel \ref{ekspartlos}. I eksemplet fandt vi løsningerne: $y_1(t)=e^{-t}$ og $y_2(t)=e^{3t}$. Den fuldstændige løsning kan således skrives:}\hfill \break
$$y(t)=k_1e^{-t}+k_2e^{3t}$$
\end{Example}
\hfill \break
\textbf{Tilfælde 2}\hfill \break
Hvis den karakteristiske løsning har én reel rod, $r$, så er diskriminanten som bekendt 0. Derfor har vi fra den karakteristiske ligning: \hfill \break
$$r=\frac{-a_1}{2a_2}\Leftrightarrow 2a_2r+a_1=0$$
\hfill \break
Det skal vi bruge til at vise, at udover $e^{rt}$, så er $y(t)=te^{rt}$ også en partikulær løsning til ligning \eqref{homlinandord}. Bemærk først at $y'(t)=e^{rt}+tre^{rt}$ og $y''(t)=2re^{rt}+tr^2e^{rt}$. Det sætter vi nu ind i ligning \eqref{homlinandord}: 
\hfill \break
\begin{align*}
a_2(2re^{rt}+tr^2e^{rt})+a_1(e^{rt}+tre^{rt})+a_0te^{rt}&=\\
e^{rt}(2a_2r+a_1)+te^{rt}(a_2r^2+a_1r+a_0)&=e^{rt}\cdot 0+te^{rt}\cdot 0=0
\end{align*}
\hfill \break
Så $te^{rt}$ er altså en løsning. det er nemt at se, at $e^{rt}$ og $te^{rt}$ er lineært uafhængige, og vi kan dermed skrive den fuldstændige løsning op:\hfill \break
$$y(t)=k_1e^{rt}+k_2te^{rt}$$ \hfill \break
\begin{Example}\textbf{Én reel rod}\hfill \break
\textnormal{Lad følgende ODE være givet:}\hfill \break
$$y''(t)+2y'(t)+y(t)=0$$\hfill \break
\textnormal{Den karakteristiske ligning $r^2+2r+1=0$ har kun én løsning, da deskriminanten $d=2^2-4\cdot 1\cdot 1=0$. Løsningen er $r=\frac{-2}{2\cdot 1}=-1$.}\\ \hfill \break
\textnormal{Dermed kan den fuldstændige løsning til ovenstående ODE skrives:}\hfill \break
$$y(t)=k_1e^{-t}+k_2te^{-t}$$
\end{Example}
\hfill \break
\textbf{Tilfælde 3}\hfill \break
Hvis den karakteristiske ligning har to komplekse rødder, $r_1$ og $r_2$, er diskriminanten negativ, og de to rødder er $r_1=\frac{-a_1}{2a_2}+i\frac{-a_1^2+4a_2a_0}{2a_2}$ og $r_2=\frac{-a_1}{2a_2}-i\frac{-a_1^2+4a_2a_0}{2a_2}$. For nemheds skyld betegner vi dog rødderne på følgende måde: $r_1=\alpha+i\beta$ og $r_2=\alpha-i\beta$. Ifølge sætning \ref{losprop2ordhom} er $e^{r_1t}$ og $e^{r_2t}$ løsninger til ligning (\ref{homlinandord}), og det er dermed nok at vise, at de to løsninger er lineært uafhængige. Det er let at indse, da $\beta \neq 0$. Hvis $k_1\in \mathbb{R}$ og $k_2 \in \mathbb{R}$ er arbitrære konstanter kan den fuldstændige løsning dermed skrives:\hfill \break
$$y(t)=k_1e^{(\alpha+i\beta)t}+k_2e^{(\alpha-i\beta)t}$$ \hfill \break
Vi vil imidlertid gerne finde en lettere måde at udtrykke den fuldstændige løsning på. Vi bruger Eulers formel \citep[s. 27]{JAB}, $e^{i\beta}=\cos(\beta)+i\sin(\beta) \enspace \forall \beta \in \mathbb{R}$, og regnereglen, $e^{\alpha+i\beta}=e^{\alpha}e^{i\beta}$, og skriver den fuldstændige løsning:\hfill \break
\begin{align}
y(t)&=k_1'e^{(\alpha+i\beta)t}+k_2'e^{(\alpha-i\beta)t} \notag \\
&=k_1'e^{\alpha t}(\cos(\beta t)+i\sin(\beta t))+k_2'e^{\alpha t}(\cos(\beta t)-i\sin(\beta t)) \notag \\
&={\alpha t}(k_1'\cos(\beta t)+k_1'i\sin(\beta t)+k_2'\cos(\beta t)-k_2'i\sin(\beta t)) \notag \\
&=e^{\alpha t}((k_1'+k_2')\cos(\beta t)+i(k_1'-k_2')\sin(\beta t)) \notag \\
&=e^{\alpha t}(k_1\cos(\beta t)+k_2\sin(\beta t)) 
\end{align}
hvor $k_1=k_1'+k_2'$ og $k_2=i(k_1'-k_2')$.\hfill \break
\begin{Example}\textbf{To komplekse rødder} \hfill \break
\textnormal{Betragt følgende ODE:}\hfill \break
$$y''(t)+2y'(t)+2y(t)=0$$\hfill \break
\textnormal{Den karakteristiske ligning har følgende løsninger:} \hfill \break
$$r_1=\frac{-2+\sqrt{2^2-4\cdot 1\cdot 2}}{2\cdot 1}=\frac{-2+\sqrt{-4}}{2}=\frac{-2+2i}{2}=-1+i$$
$$r_2=\frac{-2-\sqrt{2^2-4\cdot 1\cdot 2}}{2\cdot 1}=\frac{-2-\sqrt{-4}}{2}=\frac{-2-2i}{2}=-1-i$$\hfill \break
\textnormal{Dermed kan den fuldstændige løsning skrives:}\hfill \break
$$y(t)=e^{-t}(k_1\cos(t)+k_2\sin(t))$$
\end{Example}



Med kendskab til første og andenordens ODEs og løsninger dertil er man tættere på at kunne løse et system som for eksempel Lotka-Volterra's byttedyr-rovdyr model. Dette ses, da vi kan omskrive en andenordens ODE til et system af førsteordens ODEs. Ligeledes kan man gå den anden vej og omskrive et koblet (jævnfør afsnit \ref{ikke-lin.diff}) system af førsteordens ODEs til en andenordens ODE.

\begin{Example}
\textnormal{Betragt følgende andenordens ODE:}
$$y''(t) - 7y'(t) - 2y(t) = 0 $$
\begin{align*}
    y''(t) &= 2y(t) + 7y'(t)\\
\end{align*}
\textnormal{Vi giver nu funktionerne nye navne}
\begin{align*}
    y_1(t) &= y(t)\\
    y_2(t) &= y'(t) = y_1'(t)\\
    y_3(t) &= y''(t) = y_2'(t)
\end{align*}
\textnormal{Vi indser da, at vi har et system af differentialligninger på formen:}
\begin{align*}
    y_1'(t) &= y_2(t)\\
    y_2'(t) &= 2y_1(t) + 7y_2(t)
\end{align*}
\end{Example}