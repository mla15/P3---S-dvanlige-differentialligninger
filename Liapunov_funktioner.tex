\section{Lyapunovs metode}
Lyapunovs metode kan benyttes til at afgøre, hvilken form for stabilitet et ligevægtspunkt har i et system af differentialligninger. Afsnittet er baseret på \citep[s. 266-277]{Svensk}.\\
\hfill \break
Vi lader i det følgende $E(\vec{y})$ være en kontinuert differentiabel funktion af $n$ variable. Denne benyttes til at bestemme en funktion, der afbilleder fra $\mathbb{R}^n \to \mathbb{R}$. Da defineres den retningsafledte med hensyn til vektorfeltet, $\vec{f}(\vec{y})$, til at være 
\begin{equation}
\dot{E}_f(\vec{y})=\sum_{k=1}^{n} \frac{\partial E}{\partial y_k}(\vec{y}) f_k(\vec{y}),
\end{equation}
som også kan skrives som: $\dot{E}_f(\vec{y})=\nabla E(\vec{y}) \cdot \vec{f}(\vec{y}))$\\
\hfill \break
Det ses, at kædereglen giver
\begin{align}
\frac{dE(\vec{y}(t))}{dt}=\dot{E}_f(\vec{y}(t))
\end{align}
for enhver løsning $\vec{y}(t)$ til systemet $\dot{y}=\vec{f}(\vec{y})$\\
\hfill \break
Lad nu $y^*$ være et ligevægtspunkt til det autonome system af differentialligninger $\dot{y}=\vec{f}(\vec{y})$. For overskuelighedens skyld kan man antage, at $\vec{y}_0=\vec{0}$, thi man altid kan indføre $\vec{y}-\vec{y}^*$ som ny variabel. Derfor vil origo i det følgende betragtes som ligevægtspunkt for systemet omend andet er noteret.\\
\hfill \break
Dette motiverer definitionen af en Lyapunov funktion

\begin{definition}[Lyapunov funktion]\label{lyafunk}
Lad $\Omega$ være en åben mængde i $\mathbb{R}^n$, som indeholder origo. Betragt en kontinuert differentiabel skalar funktion $E: \mathbb{R}^n \rightarrow \mathbb{R}$.
$E(\vec{y})$ kaldes en Lyapunov funktion til systemet $\dot{y}=\vec{f}(\vec{y})$, hvor $\vec{f}(\vec{0})=\vec{0}$, såfremt:
\begin{enumerate}
  \item $E(\vec{y})>0 \ \ \text{for} \ \vec{y} \in \Omega \setminus \{\vec{0}\}$
  \item $\dot{E}_f(\vec{y})\leq 0 \ \ \text{for} \ \vec{y} \in \Omega$
\end{enumerate}
Hvis $\dot{E}_f(\vec{y})<0, \ \textnormal{for} \ \vec{y}\neq \vec{0}$, kaldes $E$ en streng Lyapunov funktion i $\Omega$.
  \end{definition}
  
  I det følgende vil vi introducere sætninger, der beskriver stabilitet, asymptotisk stabilitet og ustabilitet ved hjælp af Lyapunov funktioner, og til det formål vil vi indføre $\Omega_c$. Lad $\{\vec{y}\in \Omega | E(\vec{y})<c\}$ være en delmængde af $\Omega$, hvor $c>0$ er et arbitrært tal. Da vil vi betegne $\Omega_c$, som mængden af sammenhængende delmængder af $\Omega$ under restriktionen, at $\Omega_c$ indeholder origo samt, at løsningskurverne for $\dot{y}=\vec{f}(\vec{y})$ ikke kan forlade denne.  
  Disse mængder krymper imod origo, når $c$ aftager mod $0$. For at gøre dette klart betragter vi en lille kugle omkring origo med radius $\varepsilon>0$, hvor man kan antage, at $\varepsilon$ er så lille, at
  \begin{equation}
      B_{2\varepsilon}(0)\subseteq \Omega
  \end{equation}
  For et $r>0$ har man $B_r(0)\subseteq\Omega$, og fordi $0$ er et indre punkt i $\Omega$, kan man erstatte $\varepsilon$ med $\min(\varepsilon,\frac{r}{2})$. Da er det klart, at  $\overline{B_\varepsilon(0)}\subseteq\Omega$.\\
  \hfill \break
    Til ethvert $\varepsilon > 0$ kan man indføre 
  \begin{equation}
    c=\inf\{E(\vec{y}) \ | \ ||\vec{y}||=\varepsilon \} 
  \end{equation}
  Der gælder, at $c>0$ grundet vilkår 1 i Definition \ref{lyafunk}, og man kan dertil betragte mængden:
  \begin{equation}
      \Omega_c=\{ \vec{y}\in \Omega \ | E(\vec{y})<c, ||\vec{y}||<\varepsilon\}.
  \end{equation}
  Det er altså klart, at $\Omega_c$ ikke indeholder randen $||\vec{y}||=\varepsilon$. 

%\begin{definition}[]
%Lad $E(\vec{y})$ være en kontinuert differentiabel funktion af $n$ variable. Den afledte funktion af $E(\vec{y})$ med hensyn til vektorfeltet, $\dot{y}=\vec{f}(\vec{y})$, defineres som \hfill \break
%\begin{align*}
%\dot{E}_f(\vec{y})=\sum_{k=1}^{n} \frac{\partial E}{\partial y_k}(\vec{y}) f_k(\vec{y}).
%\end{align*}
%\end{definition}

%Man kan skrive $\dot{E}_f(\vec{y})$ som skalarproduktet mellem to vektorer i $\mathbb{R}^n$ ved
%\begin{align*}
%\dot{E}_f(\vec{y})=\nabla E(\vec{y})\cdot \vec{f}(\vec{y})
%\end{align*}   
%Den afledte funktion af $E$ med hensyn til det autonome system er dermed også den retningsafledte af $E$ i retningen, $\vec{f}(\vec{y})$. \\


%Lad nu $y_*$ være et ligevægtspunkt til det autonome system af differentialligninger $\dot{y}=\vec{f}(\vec{y})$. For overskuelighedens skyld, kan man antage at $\vec{y}_0=\vec{0}$, thi man altid kan indføre $\vec{y}-\vec{y}_*$ som ny variabel. Dermed forudsætter vi derfor fremover, at $\vec{f}(\vec{0})=\vec{0}$, altså, at origo er ligevægtspunkt.

%\begin{definition}[Lyapunov funktion]
%Lad $\Omega$ være en åben mængde i $\mathbb{R}^n$, som indeholder origo. Betragt en kontinuert differentiabel skalar funktion $E: \mathbb{R}^n \rightarrow \mathbb{R}$.
%$E(\vec{y})$ kaldes en Lyapunov funktion til systemet $\dot{y}=\vec{f}(\vec{y})$, hvor $\vec{f}(\vec{0})=\vec{0}$, såfremt:
%\begin{enumerate}
%  \item $E(\vec{y})>0 \ \forall \vec{y} \in \Omega \setminus \{\vec{0}\}$
%  \item $\dot{E}_f(\vec{y})\leq 0 \ \forall \vec{y} \in \Omega$
%\end{enumerate}
%For et åbent område $\Omega$ i en omegn af $\vec{y}_*=\vec{0}$
%Hvis $\dot{E}_f(\vec{y})<0, \ \textnormal{for} \ \vec{y}\neq \vec{0}$ kaldes $E$ streng Lyapunov funktion i $\Omega$.
%  \end{definition}



%Antag at $E(\vec{y})$ er en Lyapunov funktion i det åbne område $\Omega$. Som optakt til Lyapunovs sætninger for autonome systemer introduceres bestemte delmængder af $\Omega$, der defineres ved hjælp af $E(\vec{y})$.
%Lad $\{\vec{y}\in \Omega | E(\vec{y})<c\}$ være en delmængde af $\Omega$ for et tal $c$. 
%\hfill \break
%Lad $\Omega_c$ betegne den sammenhængskomponent af delmængden, der indeholder origo. løsningskurverne for $\dot{y}=f(y)$ kan ikke forlade $\Omega_c$. 
%\hfill \break
%Disse mængder krymper imod origo, når $c$ aftager mod $0$. For at gøre dette klart betragter vi en lille kugle omkring origo med radius $\varepsilon>0$. Sæt $c=\inf_{||\vec{y}||=\varepsilon} E(\vec{y})$. Der gælder, at $c>0$ grundet vilkår 1 i Definition 4.7, og dermed $\Omega_c \subseteq \{ \vec{y}\in \mathbb{R}^n |\ E(\vec{y})<c, \ ||\vec{y}||<\varepsilon \}$, men $\Omega_c$ skærer ikke randen $||\vec{y}||=\varepsilon$. 
%\hfill \break
\begin{mytheo}{Lyapunovs sætninger for Autonome systemer}{LyapunovSaet}
Lad $\vec{y^*} = \vec{0}$ være ligevægtspunkt for det autonome system
$$\dot{y}(t)=\vec{f}(\vec{y}(t))$$
og lad 
$$\dot{E}_f(\vec{y})=\nabla E(\vec{y}) \cdot \vec{f}(\vec{y})$$
være den retningsafledede af Lyapunov funktionen $E(\vec{y})$ med hensyn til $\vec{f}$.
\begin{enumerate}
  \item Hvis $E(\vec{y})$ er en Lyapunov funktion  i en åben mængde $\Omega$ i omegnen af $\vec{0}$, så er ligevægtspunktet stabilt.
  \item Hvis $E(\vec{y})$ er en streng Lyapunov funktion i en åben mængde $\Omega$ omkring $\vec{0}$, så er ligevægtspunktet asymptotisk stabilt.
\end{enumerate}
\end{mytheo}
\begin{proof}

\begin{enumerate}
  \item Vilkår 2 i Definition \ref{lyafunk} indebærer, at $E(\vec{y})$ aftager langs løsningskurverne. En løsningskurve, som starter i et punkt $\bar{y}$, hvor $E(\bar{y})<c$, kan derfor aldrig passere niveaukurven $E(\vec{y})=c$. For at kunne bevise dette, skal vi tilpasse disse observationer til definitionen af stabilitet, da der indgår kugler og ikke niveaukurver.\\
  \hfill \break
  Lad $\varepsilon>0$ være givet. Da kan vi vælge et $c$, så $\Omega_c \subseteq \{\vec{y}\in \mathbb{R}^n | \ ||\vec{y}||<\varepsilon \}$. Mængden $\Omega_c$ er åben og indeholder origo. Derfor findes et tal $\delta > 0$, hvorom der gælder, at mængden $\{\vec{y} \ |\ ||\vec{y}||<\delta\} \subseteq \Omega_c$.
  Da $E(\vec{y})$ er aftagende langs løsningskurverne gælder også, at:
  \begin{align*}
  \vec{y}(0)\in \Omega_c \Rightarrow \vec{y}(t) \in \Omega_c, \ \forall t\geq 0
  \end{align*}
  Det følger derved også, at:
  \begin{align*}
  ||\vec{y}(0)||<\delta \Rightarrow ||\vec{y}(t)|| < \varepsilon \ \forall t \geq 0
  \end{align*}
  Dette vil sige, at origo er stabilt ligevægtspunkt.
  
  \item  Definér $\Omega_c = \{\vec{y}\in \mathbb{R}^n | E(\vec{y}) < c\} \subseteq \mathbb{R}^n$. På grund af dette kan vi, ved at vælge tilstrækkeligt små $c$, antage, at  $\Omega_c$ er begrænset og, at komplementarmængden $\bar{\Omega}_c \subseteq \Omega.$ \hfill \break
  Betragt en løsning $\vec{y}(t)$ til systemet, hvorom der gælder $\vec{y}(0) \in \Omega_c$. Da $E(\vec{y})$ aftager langs løsningskurverne ved vi, at dette også betyder, at $\vec{y}(t) \in \Omega_c, \ \forall t\geq 0$. Funktionen givet ved $t \mapsto E(\vec{y}(t))$ er aftagende og nedadtil begrænset med $$E_0 := \lim_{t\to \infty} E(\vec{y}(t)) .$$ Den har altså en grænseværdi $E_0 \geq 0$ for $t \to \infty$. \\
  \hfill \break
  Da $E$ er positiv, mangler vi bare at vise, at $E_0=0$, for så følger det, at $ \lim_{t \to \infty} \vec{y}(t)=\vec{0}$. 
  Dette bevises ved modstrid: Antag at $E_0>0$. Da findes en radius $r>0$, hvorom der gælder, at:
  $$||\vec{y}||<r \Rightarrow E(\vec{y})< E_0.$$ Yderligere kan vi definere en mængde $\Delta=\{\vec{y}\in \bar{\Omega}_c \ | \ ||\vec{y}||\geq r\}$. Da gælder det, at
  \begin{align*}
    \vec{y}(t) \in \Delta, \ \forall t \geq 0
  \end{align*}
  Det ses da, at mængden $\Delta$ er kompakt. Hvis vi definérer et tal $k$ sådan, at $-k=max_{\Delta}\dot{E}_f(\vec{y})$, så er $-k<0$ og derved
  \begin{equation}\label{eq48}
  \frac{dE(\vec{y}(t))}{dt}\leq-k, \ \forall t\geq 0
  \end{equation}
  Dette betyder, at $E(\vec{y}(t)) \to -\infty$ for $t\to \infty$. Dette er en modstrid, da vi har antaget at $E_0>0$. Dermed må $E_0=0$, og $E(\vec{y}(t)) \to 0$ for $t \to \infty$, da $E(\vec{y}(t))$ er en aftagende funktion. Deraf gælder også, at $\vec{y}(t) \to \vec{0}$ for $t \to \infty$.
\end{enumerate}
\end{proof}

%Det ligger implicit i ovenstående bevis, at $\vec{y}(t) \to \vec{0} \textnormal{for} t \to \infty$, da man ellers ville kunne finde et $\varepsilon > 0$, så $\vec{y}(t)$ for hvilket, $B_{\varepsilon_0}(0)$ vil være falsk for vilkårligt store $t$

\begin{definition}[Asymptotisk stabilt område]
Et område $A \subseteq \mathbb{R}^n$, hvor 
\begin{align*}
    \vec{y}(0) \in A \Rightarrow \lim_{t\to \infty} \vec{y}(t)=\vec{0}
\end{align*}
kaldes asymptotisk stabilt. Hvis området er $A=\mathbb{R}^n$ siges origo (ligevægtspunktet) at være globalt asymptotisk stabilt.
\end{definition}

\begin{koro}{}{lyasaet442} 
 Lad $\Omega \subseteq \mathbb{R}^n$ være en åben delmængde, der indeholder $\vec{0}$ og antag, at $E$ er en streng Lyapunov funktion til systemet $\dot{y}=\vec{f}(\vec{y})$ i $\Omega$. Hvis $\Omega_c=\{\vec{y}\in \Omega | E(\vec{y})<c\}$ er opadtil begrænset og $\bar{\Omega}_c\subseteq \Omega$, så gælder, at
\begin{align*}
\vec{y}(0)\in \Omega_c \Rightarrow \vec{y}(t)\to \vec{0} \ \text{for} \ t \to \infty
\end{align*}
\end{koro}
\begin{proof}\\
Dette følger af beviset for 2. i Sætning \ref{th:LyapunovSaet}
\end{proof}
\begin{koro}{}{}
Antag at $\Omega=\mathbb{R}^n$ og, at $E$ er en streng Lyapunov funktion til systemet $\dot{y}=\vec{f}(\vec{y})$ med egenskaben, at $E(\vec{y})\to \infty$ for $||\vec{y}|| \to \infty$. Så er
\begin{align*}
    \lim_{t \to \infty} \vec{y}(t)=\vec{0}
\end{align*}
for enhver løsning til systemet.
\end{koro}
\begin{proof}\\
Da $E(\vec{y})\to \infty$ for $||\vec{y}|| \to \infty$ må mængden $\Omega_c$ være begrænset for ethvert $c$. Derudover gælder, at $\vec{y}(0) \in \Omega_c$ for et tilfældigt $c$. Jævnfør Sætning \ref{th:lyasaet442} konvergerer $\vec{y}(t)$ mod origo for $t \to \infty$
(Det er kun indirekte vist)
\end{proof}

\begin{Example}
\textnormal{Betragt systemet}
\begin{align*}
    y_1'(t)&=-3y_1(t)-2y_2(t)^2 \\
    y_2'(t)&=-2y_1(t)^2-4y_2(t)
\end{align*}
\textnormal{med origo som ligevægtspunkt. Funktionen $E(y_1,y_2)=y_1^2+y_2^2$ er streng Lyapunov funktion, da $E>0 \ \forall \vec{y} \in \mathbb{R}^2\setminus \vec{0}$, og}
\begin{align*}
    \dot{E}_f(y_1,y_2)&=\nabla E(y_1,y_2)\cdot \vec{f}(y_1,y_2)\\
    &=2y_1(-3y_1-2y_2^2)+2y_2(-2y_1^2-4y_2)\\
    &=-2(y_1(3y_1+2y_2^2)+y_2(2y_1^2+4y_2))\\
    &=-2(3y_1^2+2y_2^2y_1+2y_1^2y_2+4y_2^2)\\
    &=-2(y_1^2(2y_2+3)+y_2^2(2y_1+4))
\end{align*}
\textnormal{Dermed er $\dot{E}_f(y_1,y_2)<0$ for $y_1>-2, \ y_2 > -\frac{3}{2}$ \hfill \break
Origo er dermed et asymptotisk stabilt ligevægtspunkt.}
\end{Example}

\begin{Example}\label{fjedereks}
\textnormal{Betragt differentialligingen}
\begin{align*}
    y''(t)+cy'(t)+g(y(t))=0,
\end{align*}
\textnormal{hvor $g$ er en kontinuert funktion sådan, at $g(0)=0$, og}
\begin{align*}
    yg(y)>0, \ y\neq 0,
\end{align*}
\textnormal{samt $c$ er en positiv konstant. Denne ligning beskriver bevægelsen af et legeme, påvirket af en fjeder, langs y-aksen. funktionen $g(y)$ angiver fjederkraften, da denne er proportional med udstrækningen af fjederen, $cy'$ er dæmpningen ved friktion, der antages at være proportional med hastigheden, mens $y''$ er accelerationen af legemet.\\ \hfill \break
Denne ligning kan omskrives til et system af førsteordens differentialligninger ved at substituere $y_1=y$ og $y_2=y'$:}
\begin{align*}
    y_1'&=y_2\\
    y_2'&=-g(y_1)-cy_2
\end{align*}
\textnormal{Man kan forestille sig, at den totale energi i systemet kunne være en Lyapunov funktion. Dermed:}
\begin{align*}
    E(\vec{y})=\frac{1}{2}y_2^2+V(y_1), \ \textnormal{hvor}\ V(y)=\int_0^yg(s)ds
\end{align*}
\textnormal{Da $yg(y)>0, \ y\neq 0$ må $V(y_1)>0, \ y_1 \neq 0$. Dermed må $E(\vec{y})>0$, og første betingelse er opfyldt. Vi undersøger nu, om anden betingelse er opfyldt:}
\begin{align*}
    \dot{E}_f(\vec{y})=g(y_1)y_2+y_2(-g(y_1)-cy_2)=-cy_2^2\leq 0
\end{align*}
\textnormal{Dermed er $E$ altså en Lyapunov funktion, og origo er et stabilt ligevægtspunkt. Vi kan se, at $E$ derimod ikke er en streng Lyapunov funktion og kan derfor ikke afgøre, om origo er et asymptotisk stabilt ligevægtspunkt.}
\end{Example}
Anden del i Sætning \ref{th:LyapunovSaet} og Sætning \ref{th:lyasaet442} skal altså udvides, så det er muligt at påvise/afvise asymptotisk stabilitet i Eksempel \ref{fjedereks}.
\begin{mytheo}{Skærpelse af vilkår for asymptotisk stabilitet}{}
Lad $E(\vec{y})$ være Lyapunov funktion til systemet $\dot{y}=\vec{f}(\vec{y})$ i en åben mængde, $\Omega$, som inderholder origo, og lad $\vec{y^*}=\vec{0}$. Hvis origo er den eneste løsningskurve i $\Omega$, der er indeholdt i mængden $\{\vec{y}\in \Omega | \dot{E}_f(\vec{y})=0\}$, så er origo et asymptotisk stabilt ligevægtspunkt. Hvis $\Omega_c$ er begrænset og $\bar{\Omega}_c\subseteq \Omega$, så gælder, at
\begin{align*}
    \vec{y}(0)\in \Omega_c \Rightarrow \lim_{t \to \infty} \vec{y}(t)= \vec{0}
\end{align*}
\end{mytheo}
\begin{proof}\\
Som i beviset for 2. i Sætning \ref{th:LyapunovSaet} eksisterer en grænse
\begin{align*}
    E_0=\lim_{t\to \infty} E(\vec{y}(t)),
\end{align*}
fordi $\vec{y}(t)$ er en løsning med $\vec{y}(0)\in \Omega_c$.\\
\hfill \break
Antag  for modstrid, at $\vec{y}(t)$ ikke går mod origo for $t\to \infty$. Ifølge Bolzano Weierstrass' sætning \citep[s. 322]{Svensk}  findes en følge $(t_k)_{k=1}^\infty$, så $t_k\to \infty$ for $k\to \infty$, mens $\vec{y}(t_k)$ konvergerer mod et punkt $\bar{y}\neq \vec{0}$ i $\Omega_c$. Dette viser sig at være umuligt.\\
\hfill \break
Lad $y(s)$ betegne den løsning til systemet $\dot{y}=\vec{f}(\vec{y})$, der har begyndelsesværdi $\vec{y}(0)=\bar{y}$. Dette vil sige, at $\vec{y}(t_k)$ konvergerer mod $\vec{y}(0)$ for $k \to \infty$. Hvis vi sætter $\vec{y}_k(s)=\vec{y}(t_k+s)$, da tilfredstiller alle funktionerne $\vec{y}_k(s)$, hvor $k=1,2,3,\hdots$, og $\vec{y}(s)$ ligningen $\dot{y}=\vec{f}(\vec{y})$, og der fås 
\begin{align*}
    \vec{y}_k(0)\to \vec{y}(0) \ for \ k \to \infty.
\end{align*}
Det følger af Sætning 9 i \citep[s. 63]{Svensk}, at
\begin{align*}
    \vec{y}_k(s)\to \vec{y}(s) \ for \ k \to \infty \ \forall s \geq 0,
\end{align*}
eftersom $E(\vec{y})$ er kontinuert, så gælder, at 
\begin{align*}
    E(\vec{y}(s))=\lim_{k\to \infty}E(\vec{y}_k(s))=E_0,
\end{align*}
hvis dette differentieres med hensyn til $s$ vil 
\begin{align*}
    \dot{E}_f(\vec{y}(s))=0 \ \forall s \geq 0
\end{align*}
Dermed må løsningskurven for $\vec{y}(s)$ altså være indeholdt i $\{\vec{y}\in \Omega \ | \ \dot{E}_f(\vec{y})=0 \}$. Dette indebærer, at $\vec{y}(s)\equiv 0$, og dermed er $\bar{y}=\vec{y}(0)=0$. Da må $\vec{y}(t)$ nødvendigvis konvergere imod origo for $t \to \infty$ og ikke mod et andet punkt.
\end{proof}

\hfill \break
Det kan nu afgøres, om origo er asymptotisk stabilt ligevægtspunkt i Eksempel \ref{fjedereks}, 
da 
\begin{align*}
    \dot{E}_f(\vec{y})=0 \Leftrightarrow y_2=0.
\end{align*}
Men da retningsfeltet $(0,-g(y_1))$ på $y_1$-aksen er parallelt med $y_2$-aksen og dermed ikke er nul udenfor origo, så vil enhver løsningskurve, der starter på $y_1$-aksen også umiddelbart forlade denne. Dermed er origo asymptotisk stabilt ligevægtspunkt.\\
\hfill \break
Lige såvel som at undersøge hvorvidt et ligevægtspunkt henholdsvis er asymptotisk stabilt og stabilt, kan man undersøge om ligevægtspunktet er ustabilt.
\begin{mytheo}{Vilkår for ustabilitet}{}
Lad $\Omega\subseteq \mathbb{R}^n$ være en åben delmængde, der indeholder origo, og lad $\dot{y}=\vec{f}(\vec{y})$ være et system af differentialligninger, hvor $\vec{f}(\vec{0})=\vec{0}$. Antag nu, at der findes en åben delmængde $\Omega'\subseteq\Omega$ og en kontinuert differentiabel funktion $G(\vec{y})$ på $\Omega$ med følgende egenskaber:
\begin{enumerate}
  \item $\vec{0} \in \overline{\Omega'}$ \\
  \item $G(\vec{y})>0, \ \dot{G}_f(\vec{y})>0$ for $\vec{y}\in \Omega'$\\
  \item $G(\vec{y})=0$ i $\partial \Omega ' \cap \Omega$, hvor $\partial \Omega'$ betegner mængden af randpunkter for $\Omega'$ \\
\end{enumerate}
Så er origo et ustabilt ligevægtspunkt til systemet $\dot{y}=\vec{f}(\vec{y})$
\end{mytheo}
\begin{proof}\\
Fra definitionen af stabilitet af ligevægtspunkter er origo ustabilt ligevægtspunkt, hvis der findes et område $D_{\varepsilon}=\{\vec{y}\in \Omega' \ | \ ||\vec{y}||<\varepsilon\}$, hvorom der gælder, at der findes løsningskurver med begyndelsespunkt i en omegn af origo, der forlader $D_{\varepsilon}$. Der føres et indirekte bevis for at vise dette, og det antages derfor, at $\vec{y}$ ikke forlader $D_{\varepsilon}$.\\
\hfill \break
Lad $\vec{y}(t)$ være løsning til systemet $\dot{y}=\vec{f}(\vec{y})$ med $\vec{y}(0)\in \Omega'$. Vælg et $\eta> 0$ så $G(\vec{y}(0))>\eta$ og lad
\begin{align*}
    \Delta = \{\vec{y}\in \Omega' \ | \ ||\vec{y}||\leq \varepsilon, \ G(\vec{y})\geq \eta\}
\end{align*}
Dette område er kompakt, da det er fællesmængden mellem en kompakt og en lukket mængde. Nu indføres
\begin{align*}
    k=\min_{\Delta}\dot{G}_f(\vec{y}),
\end{align*}
hvor $k>0$. Da G er kontinuert differentiabel, antager $\dot{G}_f$ et infimum på $\Delta$, og dette vil være et minimum. Fra Ligning \eqref{eq48} gælder der, for $\vec{y} \in \Delta$, at:
\begin{align*}
    \frac{dG(\vec{y}(t))}{dt}\geq k
\end{align*}
Dermed vil der ved integration gælde, at $G(\vec{y}(t))\geq kt+G(\vec{y}(0))$, når $\vec{y}(t)\in \Delta$. Da $G$ er kontinuert, må den antage sit supremum, og da uligheden siger, at G mindst vokser lineært med $t$, følger det, at $\vec{y}(t)$ må forlade $\Delta$, og dermed også området $D_{\varepsilon}$.
\end{proof}