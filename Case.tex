\section{Analyse af Lotka-Volterra's rovdyr-byttedyr model}
Vi benytter Laplacetransformation, hvis vi har begyndelsesværdier, da problemet kan reduceres til algebra (Se sidst i afsnittet for laplacetransformen af Lotka-Volterra):
\begin{Example}
\textnormal{Lad os betragte følgende system af ODEs med begyndelsesværdier, $x(0)=1$ og $y(0)=0$.}
\begin{align}
    2x' + y' - y &= t\\
    x' + y' &= t^2
\end{align}
\textnormal{Vi tager laplacetransfomationen af alt}
\begin{align*}
    2(s X(s) - x(0)) + sY(s) - y(0) - Y(s) &= \frac{1}{s^2}\\
    s X(s) - x(0) + sY(s) - y(0) &= \frac{2}{s^3}\\
    2sX(s) + (s-1)Y(s) &= 2 + \frac{1}{s^2}\\
    sX(s) + sY(s) &= 1 + \frac{2}{s^3}
\end{align*}
\textnormal{Det er klart, at $X(s)$ nemt kan fjernes. Så det gør vi}
\begin{align*}
    2sX(s) + (s-1)Y(s) &= 2 + \frac{1}{s^2}\\
    -2(sX(s) + sY(s) &= 1 + \frac{2}{s^3})\\
    (s-1-2s)Y(s) &= \frac{1}{s^2} - \frac{4}{s^3}\\
    (-s-1)Y(s) &= \frac{s-4}{s^3}\\
    Y(s) &= \frac{4-s}{s^3(s + 1)}
\end{align*}
\textnormal{Vi omskriver nu til flere brøker (Ved ikke hvad partial fractions hedder på dansk)}
$$ \frac{4-2}{s^3(s+1)} = \frac{A}{s} + \frac{B}{s^2} + \frac{C}{s^3} + \frac{D}{s+1}$$
\textnormal{Der ganges igennem med $s^3(s+1)$}
$$ 4 - s = A s^2(s+1) + Bs(s+1) + C(s+1) + Ds^3$$
\textnormal{Kigger nu på ligningen og indser, at}
\begin{align*}
    A + D &= 0\\
    A + B &= 0\\
    B + C &= -1\\
    C &= 4
\end{align*}
\textnormal{Da får vi: $B = -5$, $A = 5$, $D = -5$}
$$Y(s) = \frac{5}{s} -\frac{5}{s^2} + \frac{4}{s^3} - \frac{5}{s+1}$$
\textnormal{Da anvender vi $\mathcal{L}^{-1}$}
$$ y(t) = 5 - 5t + 2t^2 - 5e^{-t}$$
\textnormal{vi isolerer nu X(s) og tager $\mathcal{L}^{-1}$ i den laplacetransformerede ligning 3.2}
\begin{align*}
    s(X)s - 1 + sY(s) &= \frac{2}{s^3}\\
    X(s) &= \frac{1}{s} + \frac{s}{s^4} - Y(s)\\
    x(t) &= 1 + \frac{1}{3}t^3 - (5 - 5t + 2t^2 - 5e^{-t})\\
    x(t) &= -4 + 5t-2t^2 + \frac{1}{3}t^3 + 5e^{-t}
\end{align*}

\end{Example}


For vores tilfælde kan vi se følgende:

\begin{equation*}
    \dfrac{db}{dt}(t) = (H-Ir(t)) b(t), 
\end{equation*}

\begin{equation*}
    \dfrac{dr}{dt}(t) = (Jb(t)-K) r(t),
\end{equation*}

$$b' - (H - Ir)b = 0 $$
$$r' - (Jb - K)r = 0 $$
Ved at tage laplacetransformen fås:
$$sB(s) - b(0) - (\frac{H}{s} - \frac{I}{s}R(s))B(s) = 0$$
$$sR(s) - r(0) - (\frac{J}{s}B(s) - \frac{K}{s})R(s) = 0$$
Så vi skal bruge nogle begyndelsesværdier ellers er vi fanget :O

\subsection{Logistisk vækst af Lotka-Volterra's model}

Hvis vi ser på systemet:
\begin{align*}
    b' &= hb - ibr\\
    r' &= jbr - kr
\end{align*}
Dette skal omskrives til et problem med logistisk vækst med hensyn til byttedyrspopulationen, hvor vi ved, at følgende udtryk skal indgå:
$$b' = pb(1 -  \frac{b}{k}),$$
hvor $k$ er en konstant, der angiver, hvor meget føde, der er tilgængelig for byttedyrene, og $p$ er en proportionalitetskonstant.
\hfill \break
I Lotka-Volterra's model illustrerer første led i øverste ligning byttedyrets population ganget med en fødselsratekonstant. Dette ændres til følgende:
\begin{align*}
    b' & = pb(1 -  \frac{b}{k}) - ibr\\
    &\Updownarrow\\
    b' &= (p(1 - \frac{b}{k}) - ir)b
\end{align*}
Da har vi et nyt system med logistisk vækst:
\begin{align*}
    b' &= pb(1 - \frac{b}{k}) - ibr\\
    r' &= jbr - kr
\end{align*}

