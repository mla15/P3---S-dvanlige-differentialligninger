
\newlength{\realparindent}
\newlength{\realparskip}
\setlength{\realparindent}{\parindent}
\setlength{\realparskip}{\parskip}



\begin{nopagebreak}
{
\begin{center}
    \samepage
    \begin{tabular*}{\textwidth}{@{} l @{\extracolsep{\fill}}r@{}}
        \parbox[b]{11cm}{
            {\LARGE Aalborg Universitet} \\
            {\large Det Teknisk-Naturvidenskabelige Fakultet}\\
            {\large Instituttet For Matematiske Fag}
        }
        & \includegraphics[width=4cm]{aau_logo} \\
        \hline
    \end{tabular*}
    \vspace{0.4cm}

    \begin{tabular*}{\textwidth}{@{}l@{\extracolsep{\fill}}r@{}}
        \multicolumn{2}{@{}l@{}}{
            \begin{minipage}[t]{1.0\textwidth}
                \begin{description}
                    \item[Titel:]~\\
                    Rovdyr-byttedyr
                    \vspace{0.5cm}
                \end{description}
            \end{minipage}
        }\\
        \begin{minipage}[t]{0.49\textwidth}
            \begin{description}
                \item[Tema:]~\\
                Sædvanlige differentialligninger
                \vspace{0.5cm}
                \item[Projektperiode:]~\\
                P3, 2. september - 21. december 2016
                \vspace{0.5cm}

                \item[Projektgruppe:]~\\
                Gruppe G4-107
                \vspace{0.5cm}  
                \item[Gruppemedlemmer:]~\\
                Amalie Rhode Høgh Nielsen 

                Bue Juul Poulsgaard
                
                Emil Egekvist
                
                Mikkel Højlund Larsen
                
                Nicklas Søndergaard Pedersen
                
                Terkel Haar Jakobsen
                
                Thomas Dam Petersen
             
                \vspace{0.5cm}
                \item[Vejledere:]~\\
                Jon Johnsen
                \vspace{0.5cm}
                
                \item[Sideantal:]
                ?
                \item[Bilagsantal:]
                ?
                \item[Afsluttet den:]
                19/12-2016
            \end{description}
        \end{minipage}
        &
        \fbox{
            \begin{minipage}[t]{0.45\textwidth}
                \textbf{Synopsis:}\\
                % Der skal være normale afsnits-indryk,
                % selv om teksten står i en minipage:
                \setlength{\parindent}{\realparindent}
                \setlength{\parskip}{\realparskip}
                {\small 
                    \input{synopsis}
                }
            \end{minipage}
        }
        \\
    \end{tabular*}
\end{center}
}

\tiny{Rapportens indhold er frit tilgængeligt, men offentliggørelse (med kildeangivelse) må kun ske efter aftale med forfatterne.}
\end{nopagebreak}