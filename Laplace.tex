\section{Laplace transformation}
Da vi først introducerede differentialligninger definerede vi dem på et interval. Denne ændring blev foretaget, da en funktion ikke altid er kontinuert over hele definitionsmængden, og her findes der andre løsningsmetoder, der er mere anvendelige end dem nævnt i Section/kapitel XX. I dette afsnit ønsker vi at komme ind på laplacetransformationer, da de også kan benyttes på diskontinuerte funktioner.


%\begin{definition}[Laplace Transformation]
%Givet en funktion $f(t)$, hvor $t \geq 0$, kan vi finde laplace transformationen $F$ af en given funktion $f$:
%$$F(s) = \mathcal{L} \{ f(t)\} = \int_{0}^{\infty} e^{-st}f(t)dt $$

%såfremt  det "improper" integrale konvergerer $\forall s \in \mathbb{R}$

%\end{definition}

%Ovenstående definition skal uddybes i forhold til, hvordan ikke-ordentlige integraler er defineret:

%$$\int_a^\infty f(t) dt = \lim_{b\to\infty} \int_a^b f(t) dt$$

%Ud fra $\lim_{b\to\infty}$, på det bundne interval fra $a$ til $b$, kan vi vurdere, hvorvidt det ikke-ordentlige integraler konvergerer eller divergerer.

%\begin{mytheo}[Linearitet af Lapacetranformationer]
%Hvis $a$ og $b$ er konstanter, gælder der, at:

%$$\mathcal{L}\{af(t) + bh(t)\} = a\mathcal{L}\{f(t)\} + b\mathcal{L}\{h(t)\}$$

%for $s = \{ s \in \mathbb{R}$ | $\mathcal{L}\{f(t)\}$ og $\mathcal{L}\{h(t)\}$eksisterer$\}$
%\end{mytheo}
