\chapter{Konklusion}
Denne rapport vil afslutningsvist indeholde en konklusion af problemformuleringen, som lyder:\\
\hfill \break
Hvordan bruges et system af differentialligninger til at modellere et byttedyr-rovdyr system, og hvilken indflydelse har det på systemet, hvis væksten af henholdsvis byttedyrene og rovdyrene er begrænset?\\
\hfill \break
Det er klart, at et system af differentialligninger er anvendeligt i den situation, hvor man står med en ligning, der er afhængig af løsningen til en anden ligning. Det vil sige, at differentialligningssystemer er anvendelige, når man har to ubekendte funktionsværdier, som er indbyrdes afhængige. Et problem af en sådan udformning, gør det imidlertid besværligt, at finde en eksakt løsning, men det giver mulighed for en kvalitativ analyse med udgangspunkt i det opstillede system.\\
\hfill \break
I det modificerede system ses der en klar forskel fra det ikke-modificerede system. Den markante ændring, som følge af modifikationen, er, at der forekommer asymptotiske ligevægtspunkter, hvormed de to populationer vil konvergere imod disse, når $t \to \infty$. Dermed vil det modificerede system opnå ligevægt over tid, hvorimod det ikke-modificerede aldrig vil opnå ligevægt og består i en uendelig cyklus.\\
\hfill \break
Konklusionen er dermed, at en kapacitetsbegrænsning på en eller begge af populationerne vil indebære, at populationerne opnår ligevægt over tid.