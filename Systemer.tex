\chapter{System af differentialligninger}

\section{Linære differentialligningssystemer}
For at kunne analysere Lotka-Volterra's model, skal man kende til differentialligningssystemers opbygning og forskellige variationer deraf. Vi har tidligere indført notationen for $\vec y$, og denne vil vi nu videreføre i følgende kapitel, hvor $\vec y$ benyttes i forbindelse med systemer af differentialligninger. 

\begin{definition}[Lineært differentialligningssystem]\label{LinSys}
Et lineært differentialligningssystem, er et system, der kan skrives på formen
$$\dot{y} = A \vec{y}$$
hvor $y \in \mathbb{R}^n$ og $A$ er en $n\times n$ matrix og
$$\dot{y} = \frac{d\dot{y}}{dt} = 
\begin{bmatrix}
\frac{dy_1}{dt} \\
\frac{dy_2}{dt}\\
\vdots \\
\frac{dy_n}{dt}
\end{bmatrix}
=
\begin{bmatrix}
a_{11}(t)y_1+a_{12}(t)y_2+ \hdots + a_{1n}(t)y_n\\
a_{21}(t)y_1+a_{22}(t)y_2+ \hdots + a_{2n(t)}y_n\\
\vdots \\
a_{n1}(t)y_1+a_{n2}(t)y_2+ \hdots + a_{nn}(t)y_n
\end{bmatrix}$$
\end{definition}
\hfill \break
Givet et IVP: $y(0) = y_0$, kan vi finde en løsning til det lineære ligningssystem, som vist ovenfor, givet ved:
$$y(t) = e^{At}y_0$$
Ved at se på definitionen for lineære differentialligningssystemer, kan det udledes, at $e^{AT}$ er en $n\times n$ matrix, som kan findes ved hjælp af taylorpolynomier.%(forklaring? xD Hvordan kommer vi egentlig frem til $e^{AT}$?) Smid en reference til prop: 2.4.4.

\subsection{Afkoblede og koblede lineære systemer}
Der findes indenfor differentialligningssystemer både afkoblede og koblede systemer. Sådanne systemer defineres nu.
\begin{definition}[Afkoblet lineært system]
Et afkoblet lineært system skrives på formen
$$\dot{y}=Ay,$$ hvor differentialligningerne kun afhænger af en variabel. Koefficientmatricen for et sådant system, er en diagonal matrix. Hvis det ikke er et afkoblet lineært system, kaldes det for et koblet lineært system.
\end{definition}
Bemærk at afkoblede lineære systemer kan løses ved hjælp af separation af variable(\ref{th:LSD}).

\begin{Example}
\textnormal{ \hfill \break
Betragt det afkoblede lineære differentialligningssystem:}
\hfill \break
\begin{align*}
    \dot{y_1}(t) &= 2y_1\\
    \dot{y_2}(t) &= y_2\\
    \dot{y_3}(t) &= -y_3
\end{align*}
\textnormal{Da kan vi skrive systemet som i \ref{LinSys}, hvor}
\hfill \break
\[ A =
\begin{bmatrix}
2 & 0 & 0\\
0 & 1 & 0\\
0 & 0 & -1\\
\end{bmatrix},
\]
\textnormal{hvormed systemet er afkoblet, hvorfor separation af variabler kan benyttes til at finde en løsning}
\begin{align*}
    y_1(t) &= c_1e^{2t}\\
    y_2(t) &= c_2e^t\\
    y_3(t) &= c_3e^-t
\end{align*}

\end{Example}

\begin{Example}\textnormal{
\hfill \break
Betragt det koblede lineære differentialligningssystem:}
\begin{align*}
    \dot{y_1}(t) &= -y_1-y_3\\
    \dot{y_2}(t) &= 4y_1-y_2-3y_3\\
    \dot{y_3}(t) &= 2y_1-4y_3
\end{align*}
\textnormal{Da kan vi skrive systemet som i \ref{LinSys}, hvor}
\hfill \break
\[ A =
\begin{bmatrix}
-1 & 0 & -1\\
4 & -1 & -3\\
2 & 0 & -4\\
\end{bmatrix},
\]
\textnormal{Et koblet differentialligninssystem kan løses vha. Laplace transformation?????????????} 
\end{Example}

\subsection{Den Fundamentale sætning for lineære systemer}

For et lineært systems koefficientmatrix, A, gælder:
\begin{lemma}{}{}
Lad $A$ være en kvadratisk matrix, så vil:
$$\frac{d}{dt}e^{At} = Ae^{At}$$
\end{lemma}

\begin{mytheo}{Den fundamentale sætning for lineære systemer}{}
Lad $A$ være en $n$ x $n$ matrix. Så vil der for et givet $y_0 \in \mathbb{R}^n$ eksistere en unik løsning til det associerede IVP ved:
$$y(t) = e^{At}y_0$$
\end{mytheo}

\section{Ikke-lineære differentialligningssystemer}

Et lineært differentialligningssystem, på formen $$\vec{y}=Ay,$$ har som sagt en unik løsning til ethvert $y_0 \in \mathbb{R}^n$. Denne løsning er givet ved $y(t)=e^{At}y_0, \ \forall t \in \mathbb{R}$. Det kan altså konkluderes at Lotka-Volterra's model ikke er et lineært differentialligningssystem. Dertil defineres nu ikke-lineære differentialligningssystemer.

\begin{definition}[Ikke-lineært differentialligningssystem] \label{ikke-lin.diff}
Et ikke-lineært differentialligningssystem, er et system, der kan skrives på formen
$$\dot{y}=\vec{f}(t, \vec{y}),$$
hvor $f: E \to \mathbb{R}^n$ og $E$ er et åbent interval af $\mathbb{R}^n$ og
$$\dot{y} = \frac{d\dot{y}}{dt} = 
\begin{bmatrix}
\frac{dy_1}{dt} \\
\frac{dy_2}{dt}\\
\vdots \\
\frac{dy_n}{dt}
\end{bmatrix}
=
\begin{bmatrix}
f_1(t, y_1, y_2, \hdots, y_n)\\
f_2(t, y_1, y_2, \hdots, y_n)\\
\vdots \\
f_n(t, y_1, y_2, \hdots, y_n)
\end{bmatrix}$$
\end{definition}

Af definition \ref{ikke-lin.diff} fremgår det at $\dot{y}$ afhænger af $t$, hvis det ikke afhænger af $t$ kaldes et sådant system for autonomt:
\begin{definition}[Autonomt/ikke-autonomt]
Et ikke-lineært system på formen
$$$$
\end{definition}

Dette afsnit kan dermed afsluttes med at konkludere, at Lotka-Volterra's model er et autonomt, ikke-lineært differentialligningssystem. Dette er umiddelbart ikke nok til at kunne vurdere, hvorledes modellen afspejler situationen i virkeligheden.