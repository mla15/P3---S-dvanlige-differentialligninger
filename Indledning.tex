\chapter{Indledning}

Det kunne være interessant at undersøge hvordan differentialligninger benyttes til modellering af for eksempel fysiske og biologiske systemer. Intuitivt kunne man forestille sig, at i et isoleret system vil populationerne af rovdyr og byttedyr påvirke hinanden således, at hvis antallet af rovdyr stiger, vil antallet af byttedyr falde. Hvis antallet af byttedyr falder, vil rovdyrene ikke have nok at spise og dermed vil populationen af disse falde. Når populationen af rovdyr er tilstrækkeligt lav, vil antallet af byttedyr stige igen. Denne sammenhæng kunne tænkes at være cyklisk. 
Dette system kan modelleres matematisk ved brug af differentialligninger. Dette gjorde Vito Volterra og Alfred J. Lotka, da Volterra’s svigersøn, Humberto D'Ancona i 1926 undersøgte forholdet mellem rovfisk og deres bytte i Adriaterhavet. Hertil udarbejdede de Lotka-Volterra-modellen, der modellerer forholdet i antal mellem to arter i et lukket system. 
Her så Lotka og Volterra at populationen af rovdyr, $r=r(t)$, hvor $r$ er antallet af rovdyr afhængigt af tiden $t$, afhang af populationen af byttedyr, $b=b(t)$, hvor $b$ er antallet af byttedyr afhængigt af tiden. 

\hfill \break
En model for et sådan system er på formen: 

\begin{equation*}
    \dfrac{db}{dt}(t) = (A-Br(t)) b(t), 
\end{equation*}

\begin{equation*}
    \dfrac{dr}{dt}(t) = (Cb(t)-D) r(t),
\end{equation*}

\hfill \break
hvor $A$, $B$, $C$ og $D$ betegner positive konstanter. Den første ligning beskriver hvordan byttedyrspopulationen udvikler sig og den anden beskriver hvordan rovdyrspopulationen udvikler sig. Disse ligninger ses at være sædvanlige differentialligninger af første orden. 

Skriv mere her!!! Muligvis lidt mere om differentialligninger!




