\chapter{Indledning}

%I dette projekt vil differentialligningssystemer blive bearbejdet, hvor der tages udgangspunkt i en case, hvor disse anvendes. Den udvalgte case for projektet omhandler Lotka-Volterra's rovdyr-byttedyr model.

%Rovdyr-byttedyr modellen er med til at beskrive en udvikling i et isoleret system, hvor man antager, at der ikke forekommer nogle ukendte variable. Der antages altså, at systemet er en akkurat afbildning af sammenhængen mellem rovdyr og byttedyr populationen under optimale forhold. Lotka-Volterra modellen er derfor et system, der kan udvides med flere variable for at gøre systemet mere virkelighedsnært. For eksempel kan man se på det udvidede system, der tager forbehold for, flere forskellige rovdyr, der konkurrerer om det samme byttedyr, eller en begrænset fødemængde til byttedyrene. I dette projekt vil der tages udgangspunkt i sidstnævnte variation af Lotka-Volterra's model, hvor modelleringsproblemet drastisk ændrer form til at være en differentialligningssystem, der inderholder logistisk vækst som en hovedingrediens (hvilken rolle den spiller kan uddybes senere).

%\textbf{Ovenstående er et alternativt forslag til midlertidig indledning :P (Systemet skal indsættes deroppe et sted)}


Dette projekt omhandler hvordan differentialligninger benyttes til modellering af fysiske og biologiske systemer. Intuitivt kunne man forestille sig, at i et isoleret system vil populationerne af rovdyr og byttedyr påvirke hinanden således, at hvis antallet af rovdyr stiger, vil antallet af byttedyr falde. Hvis antallet af byttedyr falder, vil rovdyrene ikke have nok at spise og dermed vil populationen af disse falde. Når populationen af rovdyr er tilstrækkeligt lav, vil antallet af byttedyr stige igen. 
Dette system kan modelleres matematisk ved brug af differentialligninger. Dette gjorde Vito Volterra og Alfred J. Lotka uafhængigt af hinanden i starten af 1900-tallet \citep{wikiLV}. De udarbejdede følgende model, som senere blev kendt som Lotka-Volterra-modellen:

\begin{equation}\label{byttedyr}
    \dfrac{db}{dt}(t) = (A-Br(t)) b(t), 
\end{equation}

\begin{equation}\label{rovdyr}
    \dfrac{dr}{dt}(t) = (Cb(t)-D) r(t),
\end{equation}

\hfill \break
hvor $A$, $B$, $C$ og $D$ betegner positive konstanter. Den første ligning beskriver, hvordan byttedyrspopulationen udvikler sig over tid, og den anden beskriver, hvordan rovdyrspopulationen udvikler sig over tid.\\ \hfill \break
Funktionen $b(t)$ er et udtryk for hvor mange byttedyr, der er til tiden $t$. Ligeledes er $r(t)$ et udtryk for hvor mange rovdyr, der er til tiden $t$. Det ses, at konstanten $A$ ganges på $b(t)$ i ligning \eqref{byttedyr}. Det vil sige at $A$ er et udtryk for hvor hurtigt byttedyrsbestanden vil stige uafhængigt af rovdyrsbestanden. Med andre ord, er det et udtryk for, i hvilken grad byttedyrene reproducerer. Konstanten $B$ ganges med $b(t)$ og $r(t)$, og er et udtryk for, i hvilken grad byttedyr bliver ædt af rovdyr. Der bliver således ædt flest byttedyr, hvis der både er mange rovdyr og mange byttedyr, hvilket intuitivt giver god mening. Betragtes hele ligning \eqref{byttedyr}, ses det, at ændringen i antallet af byttedyr er lig med antallet af nye byttedyr, $Ab(t)$, minus antallet af byttedyr, som bliver ædt, $Br(t)b(t)$.\\ \hfill \break
I ligning \eqref{rovdyr} ses det, at leddet som beskriver tilvæksten i rovdyr er $Cb(t)r(t)$. Dermed er $C$ et udtryk for, i hvilken grad rovdyrene reproducerer. I modellen antages det altså, at rovdyrenes reproduktion er proportional med antallet af byttedyr. Konstanten $D$ ganges på $r(t)$ og er et udtryk for hvor mange rovdyr der dør. Antallet af rovdyr, der dør, er således højere jo flere rovdyr, der er. \\ \hfill \break
I modellen er der taget en række antagelser om systemet, for at kunne modellere det. Disse antagelser er \citep{wikiLV}:
\begin{enumerate}
    \item Byttedyrene finder altid rigeligt føde.
    \item Rovdyrenes forsyning af føde afhænger udelukkende af størrelsen af bytterdyrspopulationen.
    \item Hastigheden hvormed en populations størrelse ændrer sig er proportional med populationens størrelse.
    \item Miljøet ændrer sig ikke til fordel for en af populationerne, og genetisk tilpasning har ingen konsekvenser.
    \item Rovdyr har uendelig appetit.
\end{enumerate}
I dette projekt beskæftiger vi os ikke med punkterne 2-5, men vi vil undersøge, hvilken indflydelse det har på systemet, hvis mængden af føde til byttedyrene er begrænset. Derudover vil vi beskrive systemet i ligninger \eqref{byttedyr} og \eqref{rovdyr} i lyset af teorien om differentialligninger.
\hfill \break
\section{Problemformulering} 
Hvordan bruges et system af differentialligninger til at modellere et byttedyr-rovdyr system, og hvilken indflydelse har det på systemet, hvis mængden af føde til byttedyrene er begrænset?
%Hvordan vil en begrænset mængde føde til byttedyrene influere systemets stabilitet?


